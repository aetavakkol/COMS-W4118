\documentclass[]{article}
\usepackage[all]{xy}
\usepackage{amsmath}
\usepackage{amssymb}
\usepackage{amsthm}
\usepackage{enumitem}
\usepackage{indentfirst}
\usepackage{listings}
\usepackage{multirow}
\usepackage{tikz}
\usepackage{tikz-qtree}
\usepackage{tipa}
\begin{document}

\newtheorem{thm}{Theorem}
\title{Operating Systems \\ COMS W4118 \\ Lecture 21}
\author{Alexander Roth}
\date{2015 -- 04 -- 14}
\maketitle

\section{Disk Interface}
\begin{itemize}
\item For the file system, the disk is just a one dimensional array of logical
sectors.
\item Modern drives export logical block addresses and do the mapping ot the
cylinder/head/sector internally.
\item If you try to read sequentially through a block, the head will read
through the platter and move inwards towards the center of the disk.
\item The disk is optimized for sequential access and sequential access is fast.
\item It is a good idea to arrange information sequentially.
\end{itemize}

\section{File Systems}
\begin{itemize}
\item There are two modes of access:
\begin{enumerate}
\item Sequential Access
\item Random Access
\begin{itemize}
\item Randomly address any block, difficult to make fast.
\end{itemize}
\end{enumerate}
\item We want to minimize unnecessary slow down by modeling the data system
cleverly.
\end{itemize}

\subsection{Disk Management}
\begin{itemize}
\item We need to keep track of where the file is and which file takes up which
block.
\item Need to track where the file is on disk.
\item Need to have some metadata on disk that maintains the information of the
disk, which is handled by the file system.
\end{itemize}

\subsection{Allocation Strategies}
\begin{itemize}
\item There are various approaches to allocating information on a disk.
\end{itemize}

\subsubsection{Contiguous Allocation}
\begin{itemize}
\item We will allocate a contiguous number of blocks to a file and break it up
accordingly.
\item You need to keep track of where a file is and how big it goes.
\item Most of the block is used for itself.
\item easy to implement, there is a low storage overhead.
\item The issue is that it is almost impossible to grow the file.
\item There can be overlaps in the file, so we may need to relocate the file
accordingly.
\item If files increase in size and need to be relocated, there will be holes
within the disk, where information cannot sit. Thus, there is \emph{external
fragmentation}.
\item \emph{Internal fragmentation} is the issue of not all memory being written
within a block.
\item These systems are typically used for write-once read-often systems. Such
as DVD's, tape drives.
\end{itemize}

\subsubsection{Extent-based Allocation}
\begin{itemize}
\item Instead of one contiguous chunk per file, there are multiple chunks per
file.
\item Similar to contiguous allocation, but with multiple chunks.
\item The metadata has to keep track of more information: start of the chunk and
length of the chunks.
\end{itemize}

\subsubsection{Linked Allocation}
\begin{itemize}
\item Treat the disk drive as a linked list.
\item There is a chunk of data within the block, and the last segment of the
data is a pointer to the next block in the chain.
\end{itemize}

\subsubsection{FAT Table}
\begin{itemize}
\item There is a file allocation table that contains pointers to the addresses
of all the files on the disk.
\item The files on the disk have pointers to the other chunks of the same file.
\item All the file table knows is that there is a file at a given address.
\item Fast random access. Only search through the cached FAT.
\item There is a large storage overhead for FAT table.
\item Possibly slow sequential access.
\end{itemize}

\subsubsection{Indexed Allocation}
\begin{itemize}
\item You store in a file metadata all the blocks that a file has.
\item The first block holds all the indexes to the other chunks.
\item Unix systems use an inode
\end{itemize}

\end{document}
