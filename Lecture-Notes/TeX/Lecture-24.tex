\documentclass[]{article}
\usepackage[all]{xy}
\usepackage{amsmath}
\usepackage{amssymb}
\usepackage{amsthm}
\usepackage{enumitem}
\usepackage{indentfirst}
\usepackage{listings}
\usepackage{multirow}
\usepackage{tikz}
\usepackage{tikz-qtree}
\usepackage{tipa}
\begin{document}

\newtheorem{thm}{Theorem}
\title{Operating Systems \\ COMS W4118 \\ Lecture 24}
\author{Alexander Roth}
\date{2015 -- 04 -- 23}
\maketitle

\section{Scheduling II}
\subsection{Motivation}
\begin{itemize}
\item There is no one-size-fits-all scheduling algorithm.
\item The Completely Fair Scheduler is a complicated beast.
\item Many real scheduling algorithms combine many elements of the simple algorithms that we see.
\end{itemize}

\subsection{Combining Scheduling Algorithms}
\begin{itemize}
\item The conceptual image is there are a list of running processes.
\item The kernel picks one process to run next.
\item The real picture is more complicated.
\item There are multi-level queues.
\item We have multiple queues because most systems have multiple CPUs.
\item Each queue can have it's own scheduling algorithm.
\item Foreground processes should have higher priority than the background processes.
\end{itemize}

\subsection{Global Queue of Processes}
\begin{itemize}
\item There is one ready queue shared across all CPUs
\end{itemize}

\subsection{Per-CPU Queue Of Processes}
\begin{itemize}
\item We statically partition processes to the CPU.
\item Easy to implement, but there is the issue of a load imbalance.
\end{itemize}

\subsection{Hybrid Approach}
\begin{itemize}
\item Use a hybrid approach. Have a global queue and a per-CPU queue.
\item Ability to move processes around.
\end{itemize}

\subsection{SMP: ``Gang'' Scheduling}
\begin{itemize}
\item If you have multiple squads that work together, producer-consumer model.
\item These systems will work best if everything is working at the same time.
\item The operating system will try to schedule the system together as much as possible.
\end{itemize}

\subsection{Real-Time Scheduling}
\begin{itemize}
\item Linux supports Real-Time Scheduling
\item There are two kinds of real-time scheduling.
\begin{enumerate}
\item Hard Real-Time, and
\item Soft Real-Time
\end{enumerate}
\item Soft Real-Time: If you have a task that is designated as real-time, it will run and preempt any normal processes.
\item Hard Real-Time is difficult to implement because of all the processes that are running.
\end{itemize}

\subsection{Linux Scheduling Overiew}
\begin{itemize}
\item There are multiple levels of queues with priority
\item There are hundreds of levels of priority.
\item These are hard priority levels, so a higher priority preempts the lower level.
\item 0 is the normal priority.
\item There are sublevels of priorities in a hard level.
\item Linux has 100 static priority levels.
\item Nice values are the priority values within a level.
\end{itemize}

\subsection{Completely Fair Scheduler (CFS)}
\begin{itemize}
\item Calculates schedule latency: the amount of time it takes for exch process to run at least once.
\item Everyone has to run the schedule latency.
\item CFS is not completely fair if there are more processes than you can fit in the latency period in a reasonable fashion.
\end{itemize}

\end{document}
