\documentclass[]{article}
\usepackage[all]{xy}
\usepackage{amsmath}
\usepackage{amssymb}
\usepackage{amsthm}
\usepackage{enumitem}
\usepackage{indentfirst}
\usepackage{listings}
\usepackage{multirow}
\usepackage{tikz}
\usepackage{tikz-qtree}
\usepackage{tipa}
\begin{document}

\newcommand{\code}{\texttt}
\newtheorem{thm}{Theorem}
\title{Operating Systems \\ COMS W4118 \\ Lecture 19}
\author{Alexander Roth}
\date{2015 -- 04 -- 07}
\maketitle

\section{Diving into the Linux Kernel}
\begin{itemize}
\item the macro \code{SYSCALL\_DEFINE0} is used to define a system call with no
formal parameters.
\item \code{pid} is really the id of the \code{task\_struct} which is used to
denote the value of each thread.
\item There is some virtualization built into the linux kernel.
\item Linux has this concept of a \emph{container}.
\item A container is a small world within the kernel that allows for
virtualization throughout the system.
\item Self-contained environments for processes.
\item It is possible that using a container creates different PID's from what
the user sees and what the kernel actually knows about.
\item These different PID's are mapped together so the kernel knows what the
user is referring to.
\item There are two structs in the \code{task\_struct}: parent and real parent.
\item Real parent is the process that created you. Parent is the process that is
the current parent, or the process that receives the \code{SIGCHILD} signal.
\item The distinction between \code{getppid} and \code{getpid} is the use of
locking.
\item We create a readlock for \code{getppid} in order to avoid the issue where
the parent could exit by the time the function returns.
\item \code{rcu\_read\_lock} is very similar to a reader/writer lock, but this
goes a step further.
\item It is known as read copy update lock. You can have multiple readers and
one writer acting at the same time.
\item When a writer wants to update a data structure, it makes a copy of the
structure.
\item The structure is accessed through only one pointer.
\item Once the write is complete, the writer switches the pointer to the old
copy to the new copy atomically.
\item It is possible in rcu for the reader to read an older version of a piece
of data.
\item However, the reader will never read an incomplete version of the data.
\item There is a mechanism to make sure that writer does not change values until
all the readers complete their read of the old data page.
\item There is a significant overhead by creating a reader lock every time a
reader must access a variable.
\item You may have to go to the top of the file to see what's going on.
\item On a certain level, you need to make an abstraction and abstract some of
the lower level detail.
\item You need to learn how to cut some things and abstract systems away.
\item You must assume some certain operations.
\item You must learn how to look at things and assume that's how they work.
\item Locking should be avoided as much as possible.
\item If you do not need to lock the majority of cases, then you should avoid
locking and find another way to correct it.
\end{itemize}

\section{Implementing Locks}
\begin{itemize}
\item What does it mean to yield?
\item What does it mane to sleep?
\item The big question is ``What does it mean for a process to sleep?''
\item The virtual memory space is made up of 4G.
\item With the typical structure of text, followed by static, followed by the
heap.
\item Where the heap begins is called the \emph{brake} or \code{brk}.
\item The stack starts around 3G and maps down, with empty space between the
heap and the stack.
\item The last 1G is the kernel code. \code{bzImage} file is the kernel code.
\item The static section of the memory along with the text memory is mapped to
physical regions within the RAM of the system.
\item The OS silently converts these areas of code from virtual address space to
physical address space.
\item These virtual addresses must be transformed in order for the machine to be
able to function correctly.
\item The kernel image and data sections are common to all processes.
\item Thus, they are images of a single piece of the memory.
\item Each process only has one copy of the kernel running.
\item The kernel memory contains the \code{bzImage} and a stack.
\item It must have a stack for each process, which is typically 8K for each
process.
\item The starting and ending address for each process falls on the 8K boundary.
\item You would take away the last 13 bits to locate the \code{thread\_info}
struct for the current task.
\item The \code{thread\_info} has a pointer to the \code{task\_struct}.
\item Where are the task structs?
\item Within the kernel space, there must be task structs for everyone.
\item All the task structs are linked together through an embedded linked list
in the \code{list\_head} struct. This is a doubly linked list.
\item The very first process is \code{init\_task} with a \code{pid} of 0, known
as the swapper thread.
\item Within the \code{struct task\_list} there is a \code{int} variable known
as \code{state}.
\item \code{state} can have a number of values.
\begin{enumerate}
\item \code{TASK\_RUNNING}
\item \code{TASK\_INTERUPTABLE}
\item \code{TASK\_UNINTERUPTABLE}
\item \code{TASK\_TRACED}
\item \code{TASK\_ZOMBIE}
\end{enumerate}
\item Linux uses \code{TASK\_RUNNING} for running processes and processes that
can be run.
\item \code{TASK\_INTERUPTABLE} and \code{TASK\_UNINTERUPTABLE} are used for
sleeping processes.
\item The scheduler moves through the linked list and gets the next runnable
task on the run queue.
\item The run queue now is a $n$-something list with a method to set priorities.
\end{itemize}

\end{document}
