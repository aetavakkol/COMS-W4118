\documentclass[]{article}
\usepackage[all]{xy}
\usepackage{amsmath}
\usepackage{amssymb}
\usepackage{amsthm}
\usepackage{enumitem}
\usepackage{indentfirst}
\usepackage{listings}
\usepackage{multirow}
\usepackage{tikz}
\usepackage{tikz-qtree}
\usepackage{tipa}
\begin{document}

\newcommand{\code}{\texttt}
\newtheorem{thm}{Theorem}
\title{Operating Systems \\ COMS W4118 \\ Lecture 12}
\author{Alexander Roth}
\date{2015 -- 03 -- 03}
\maketitle

\section{UNIX Domain Sockets}
\begin{itemize}
\item Reliable across datatype modes.
\end{itemize}

\subsection{File Sharing}
\begin{itemize}
\item When you for \code{fork} a process, you \code{dup} all the file
descriptors into the child as well as the parent.
\item Sometimes it is useful to pass an open file between processes, but you
can't just pass the values to another process with pipes, etc.
\item Domain socket is the only facility that allows you to pass an open file
descriptor to another process.
\item Threads share the same memory, so they are able to access the same file.
\end{itemize}

\section{Single and Multithreaded Processes}
\subsection{Why Use Threads?}
\begin{itemize}
\item Express concurrency
\item Efficient communication
\item Leverage multiple cores (depends)
\item Having multithreaded programs allows for easier implementation.
\item Multithreading simplifies single threading logic by having separate
threads that handle tasks.
\item A good example is a multiplayer game server.
\item There is no easy way to control scheduling in a single thread.
\item The problem with shared memory is that use my setup the region, designate
the reason, and tear down the region.
\end{itemize}

\subsection{Multithreading Models}
\begin{itemize}
\item User threads: thread management done by user-level threads library; kernel
knows nothing
\item Kernel threads: threads directly supported by the kernel.
\item The kernel knows about these threads.
\item Kernel level threads are aware of the threads and there is a threaded
table that is a link list of \code{task struct}s.
\item If you create a thread in \code{pthread\_create}, linux creates a clone of
the thread and schedules it accordingly.
\item Another way to implement threading is to cheat and not let the kernel know
about the thread and do everything in the user level process. Do everything in
the user space.
\item Kernel threads are just like processes, they are scheduled by the kernel.
\item User threads cannot be preemptively scheduled.
\item In a program, you must call \code{pthread\_yield()}, which will save the
thread in some position in memory and jump to a different routine.
\end{itemize}

\subsection{User Thread Blocking}
\begin{itemize}
\item One block thread can block all the threads because the kernel just sees
one read and does not know that there are other threads in the program.
\item We can manage this by using \code{select()} to intercept every system call
beforehand.
\item This is a possible solution; however, it's messy and hard to implement.
\end{itemize}

\subsection{Scheduler Activations}
\begin{itemize}
\item User-level threads are not really used anymore.
\item However, the kernel is somewhat aware of the threading.
\item When a blocking method is call, the call will go to the kernel
\item Since the kernel is away of the threads, it will block all the threads or
none at all.
\item The kernel will alert the thread scheduler and let's it know it can run
other threads in the mean time.
\item When the blocking call completes, it will send an interrupt to interrupt
the current thread that is active.
\item The kernel then alerts the thread scheduler that the blocking call is
complete.
\item The kernel just knows about the thread scheduler.
\item Slightly violates the encapsulation and separation between the kernel and
the user space because the kernel needs to know about user threads.
\end{itemize}

\end{document}
