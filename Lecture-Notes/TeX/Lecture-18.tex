\documentclass[]{article}
\usepackage[all]{xy}
\usepackage{amsmath}
\usepackage{amssymb}
\usepackage{amsthm}
\usepackage{enumitem}
\usepackage{indentfirst}
\usepackage{listings}
\usepackage{multirow}
\usepackage{tikz}
\usepackage{tikz-qtree}
\usepackage{tipa}
\begin{document}

\newcommand{\code}{\texttt}
\newtheorem{thm}{Theorem}
\title{Operating Systems \\ COMS W4118 \\ Lecture 18}
\author{Alexander Roth}
\date{2015 -- 04 -- 02}
\maketitle

\section{The Linux Kernel}
\subsection{Header Files}
\begin{itemize}
\item It's impossible to know everything about the linux kernel at any given
time.
\item It's just too big. \emph{Phrasing}.
\item You are not going to be able to become a code ninja over night.
\end{itemize}

\subsection{Source Code}
\begin{itemize}
\item The documentation directory holds all necessary documentation.
\item The \code{kernel} file is the top level directory.
\end{itemize}

\subsection{Processes or Threads}
\begin{itemize}
\item The generic term in linux is \code{task}
\item In theoretical operating systems, the names are usually process
descriptor, process control block, task descriptor, etc.
\item These things are some structure in the kernel memory that maintain a link
list that represents running processes.
\item If you have multiple threads in a process, they all share these resources.
\item \code{task struct} maintain pointers to all the necessary values in the
operating system.
\item When a user program calls read, the system jumps into kernel mode.
\item When we are in kernel mode, there is a special kernel stack that is
created.
\item The kernel cannot trust addresses provided by the user.
\item It is not a privileged operation to load things into a register.
\item The kernel is in complete control and only one copy of the kernel is
running at a time, so you only need one kernel stack.
\item Once you are in kernel mode, you are never going to switch processes.
\item When linux had to support multiple CPU's, a locking mechanism was needed.
\item A pre-empted kernel means that the kernel is safe to run on multiple CPU's
at the same time.
\item Every process has two stacks
\begin{enumerate}
\item User stack
\item Small part of the kernel stack (typically 4KB)
\end{enumerate}
\item Memory is counted in chunks or pages. One page is typically 4KB.
\end{itemize}

\subsection{Process Descriptor}
\begin{itemize}
\item \code{task\_struct} is a huge structure.
\item If you are running something in \code{sudo}, the effective user becomes
\code{boot}, while the standard user is yourself.
\item Effective user determines what you can and cannot do within a system.
\item Linux uses part of a task's kernel-stack page-frame to store thread 
information.
\item Do not use the word current because it's a defined macro.
\item Processes are all interlinked in an embedded linked list.
\end{itemize}

\end{document}
