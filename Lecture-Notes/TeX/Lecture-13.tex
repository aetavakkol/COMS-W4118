\documentclass[]{article}
\usepackage[all]{xy}
\usepackage{amsmath}
\usepackage{amssymb}
\usepackage{amsthm}
\usepackage{enumitem}
\usepackage{indentfirst}
\usepackage{listings}
\usepackage{multirow}
\usepackage{tikz}
\usepackage{tikz-qtree}
\usepackage{tipa}
\begin{document}

\newcommand{\code}{\texttt}
\newtheorem{thm}{Theorem}
\title{Operating Systems \\ COMS W4118 \\ Lecture 13}
\author{Alexander Roth}
\date{2015 -- 03 -- 05}
\maketitle

\section{Thread Design Issues}
\begin{itemize}
\item The underlying kernel will choose any random thread to run a program.
\item The thread is chosen from any underlying thread that is willing to accept
a signal.
\end{itemize}

\section{Synchronization}
\begin{itemize}
\item \code{++} and \code{--} are not thread-safe operators.
\item \code{taskset} is a program you can use that runs other programs while
setting the specific CPU to use.
\end{itemize}

\subsection{Race Conditions}
\begin{itemize}
\item Hard to reproduce unless you are using multiple CPU's.
\item Non-deterministic.
\item We can attempt to fix the scheduling of the operating system, so that the
same thing happens every time.
\item If you can reproduce the exact scheduling from time 0 to some point, there
is a good chance you can reproduce the error.
\end{itemize}

\subsection{How to Avoid Race Conditions?}
\begin{itemize}
\item We can use locking.
\item Atomic operations (all or nothing) would prevent operations from being
interleaved.
\item Create a super instruction that does what we want atomically.
\item We cannot have an atomic instruction for every single possible case.
\item \textbf{Critical sections} are sections of code that cannot be broken up.
They must be atomic.
\item \textbf{Liveness} cannot allow a deadlock to occur.
\item We strive for \textbf{Safety}, \textbf{Liveness}, and \textbf{Bounded
waiting} from our threading environment.
\item It is possible for an operating system to not schedule one thread for a
while.
\item An operating system may not schedule threads in a fair, distributed
fashion.
\item Thus, the synchronization method must work for any scenario, even if the
thread is running slowly.
\item Critical sections should be \textbf{efficient}, \textbf{fair}, and
\textbf{simple}.
\item Busy waiting (or spin locking) happens often in the kernel. Look it up.
\item Efficiency must be considered in context. There is not one absolute case.
\item Fairness implies that a thread should not wait too long to run.
\item Complex API's are always doomed for failure, so be sure to avoid this
issue.
\end{itemize}

\subsection{Implementing Critical Sections}
\subsubsection{Version 1: Disable Interrupts}
\begin{itemize}
\item Concurrency issues happen when signals interrupt threads.
\item Timer interrupts cause processes to halt and jump between processes.
\item Disabling interrupts only makes sense in the kernel, not the user space.
\item If you disable signal interrupt on a single CPU, the running program will
run forever.
\item Disabling interrupts allows a set of operations to become atomic.
\item This style of coding does not work on multiprocessors.
\item Disabling interrupts ruins efficiency on the system, so it's generally a
bad idea.
\item This is legitimate style on uni-processing systems, typically when the
kernel needed to lock something and do something.
\end{itemize}

\subsubsection{Version 2: Software Locks}
\begin{itemize}
\item There are a number of software algorithms that we can implement to provide
a locking mechanism.
\item The \textbf{Peterson's algorithm} is a software-based lock algorithm that
does not require anything special from the hardware.
\item The only assumptions needed are:
\begin{enumerate}
\item that loads and stores must be atomic.
\item We assume hardware operations execute in a serial order.
\item We do not require special hardware instructions.
\end{enumerate}
\item In general adversarial scheduler model useful to think about concurrency
problems.
\end{itemize}

\end{document}
