\documentclass[]{article}
\usepackage[all]{xy}
\usepackage{amsmath}
\usepackage{amssymb}
\usepackage{amsthm}
\usepackage{enumitem}
\usepackage{indentfirst}
\usepackage{listings}
\usepackage{multirow}
\usepackage{tikz}
\usepackage{tikz-qtree}
\usepackage{tipa}
\begin{document}

\newcommand{\code}{\texttt}
\newtheorem{thm}{Theorem}
\title{Operating Systems \\ COMS W4118 \\ Lecture 8}
\author{Alexander Roth}
\date{2015 -- 02 -- 17}
\maketitle

\section{Signal API}
\begin{itemize}
\item \code{sigset\_t} is an integer that is assigned to a signal.
\item There is no signal number 0.
\item If you have more than 31 signals, you cannot use an integer to represent
it and instead you must use a \code{long}.
\item \code{sigset\_t} is a generic name to control the number of the signal.
\item We assume that it is an \code{int}.
\item The operating system will queue a pending signal, but it will only queue
one of the signals.
\item One of each signal can be pending.
\end{itemize}

\section{\code{sigaction()} function}
\begin{itemize}
\item While in a signal handler, you can tell the operating system to block
certain signals.
\item \code{sigaction} always retains the signal handler. So it will not revert
to the default value after 1 signal is caught.
\end{itemize}

\section{POSIX Threads}
\begin{itemize}
\item Threading is an extremely important concept for Operating Systems.
\item When you \code{fork}, you create a child and a parent process.
\item Web servers should \code{fork} when \code{accept()} is called to allow
multiple clients in the system.
\item In order to allow process communication, you would need shared memory or
pipes between the client processes.
\item Threading allows two processes to be running, but they share the same
memory space.
\item Spawn a thread and it shares a memory space with other threads.
\item Threads do not share a stack.
\item Each thread has a placement on the thread.
\item \code{pthread\_join} is similar to \code{wait\_pid}.
\end{itemize}

\end{document}
