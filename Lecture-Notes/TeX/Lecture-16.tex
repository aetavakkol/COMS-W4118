\documentclass[]{article}
\usepackage[all]{xy}
\usepackage{amsmath}
\usepackage{amssymb}
\usepackage{amsthm}
\usepackage{enumitem}
\usepackage{indentfirst}
\usepackage{listings}
\usepackage{multirow}
\usepackage{tikz}
\usepackage{tikz-qtree}
\usepackage{tipa}
\begin{document}

\newcommand{\code}{\texttt}
\newtheorem{thm}{Theorem}
\title{Operating Systems \\ COMS W4118 \\ Lecture 16}
\author{Alexander Roth}
\date{2015 -- 03 -- 24}
\maketitle

\section{Semaphore Uses: Execution Order}
\begin{itemize}
\item Semaphore allows to different threads to wait and post separately.
\item Can be used when you want to impose order.
\item Mutex just makes sure that threads cannot overlap.
\item Cannonical example is the producer-consumer model.
\end{itemize}

\section{Conditions Vs. Semaphore}
\begin{itemize}
\item Semaphores are sticky, they have a number.
\item Every thread that calls \code{sem\_post} will increment the semaphore.
\item Every thread that calls \code{sem\_wait} will decrement the semaphore.
\item There is no waiting.
\item The only time you block with a semaphore is when the semaphore goes to 0,
the number must be positive.
\item The ordering doesn't matter in terms of coordination. Things can get lost.
\item Semaphores have memory
\item Condition variables do not. They only send a signal that is expected to be
caught by a
\end{itemize}

\section{\code{sem\_post}}
\begin{itemize}
\item Locks the mutex within the semaphore.
\item Increments the count of the semaphore.
\item Alert the threads to the increment with \code{pthread\_cond\_signal}.
\item Unlock the mutex within the semaphore.
\end{itemize}

\section{\code{sem\_wait}}
\begin{itemize}
\item Lock the mutex
\item Check that the count is not zero.
\item If the count is zero set the wait, come back and check later.
\item Otherwisre, decrement the count.
\item Unlock the mutex within the semaphore.
\end{itemize}

\section{Mesa Semantics}
\begin{itemize}
\item Modern operating systems use this type of competition variables.
\item When you come out a wait, you have to reacquire the lock; otherwise
another process can come in and take this lock.
\item \textbf{Hoare Semantics:} Suspends the signaling thread, and immediately
transfers control to the woken thread.
\item \textbf{Mesa Semantics:} \code{signal()} moves a signle wiating thread
from the blocked state to a runnable state, then the signaling thread continues
until it exists the monitor.
\item Typically, you will signal before you unlock the mutex.
\end{itemize}

\section{System Calls}
\begin{itemize}
\item The kernel address space contains every process mapped to it.
\item Every process has the top first gigabyte mapped to the kernel.
\item Virtual address spaces can be mapped to single physical addresses.
\item If you call a system call that is executed on behalf of a process, there
is a little bit of the stack allocated for each process.
\item A small amount of stack is allocated in the kernel for any process calls
that would appear.
\item System calls elevate privilege of user process.
\item The CPU switches into privilege mode, which allows all instructions to
become available.
\item Thus, any user can inject code into the system.
\item System calls must verify every single parameter to make sure it is not
doing anything illegal.
\item You should not be able to pass any pointer to write because you can
overwrite memory you weren't supposed to touch.
\end{itemize}

\subsection{System Call Dispatch}
\begin{itemize}
\item A user process invokes a system call.
\item There is some facility that allows the CPU to switch from the user mode to
kernel mode.
\item There are three ways that a running process can be interrupted.
\begin{enumerate}
\item Hardware Interrupt - Sent from device to processor (timers, etc.)
\item Software Interrupt - Instruction loaded by processor, you are interrupting
yourself to do something else. This is the only way to jump into the kernel and
make system calls.
\item Processor - Exceptions (invalid instructions)
\end{enumerate}
\end{itemize}


\end{document}
