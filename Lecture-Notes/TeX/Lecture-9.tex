\documentclass[]{article}
\usepackage[all]{xy}
\usepackage{amsmath}
\usepackage{amssymb}
\usepackage{amsthm}
\usepackage{enumitem}
\usepackage{indentfirst}
\usepackage{listings}
\usepackage{multirow}
\usepackage{tikz}
\usepackage{tikz-qtree}
\usepackage{tipa}
\begin{document}

\newcommand{\code}{\texttt}
\newtheorem{thm}{Theorem}
\title{Operating Systems \\ COMS W4118 \\ Lecture 9}
\author{Alexander Roth}
\date{2015 -- 02 -- 19}
\maketitle

\section{POSIX threads}
\subsection{Create, Exit, Join}
\begin{itemize}
\item \code{pthread\_t} is an implementation type that is system-dependent.
\item Linux ignores the notion of threads. Threads are just processes that share
memory spaces with the parent process.
\item In modern Linux, threads are now a different concept from processes.
\item The \code{ps} command evolved from the old days of UNIX.
\item \code{LWP} - Light Weight Process ID is the same thing as thread ID.
\item The leader thread shares the same process ID and the same thread ID.
\item Linux uses something like process ID for thread ID. It appears they come
from the same pool of numbers.
\item All threads from within a process share a process.
\end{itemize}

\subsection{Mutex}
\begin{itemize}
\item An object you declare and initialize
\item Behaves like a lock.
\item You declare a variable of the \code{mutex} type and you pass in the
attributes to it.
\item There is a timeout version of the mutex api, where if during the time the
mutex is unable to lock, then it will throw an error.
\item \code{trylock} is a non-blocking version of \code{block}.
\item Semaphore has a number and does not belong to any processes.
\item Mutexes are primarily locking mechanisms.
\item Semaphores have more ability to share locks across threads, you cannot do
that through mutexes.
\end{itemize}

\subsection{Deadlock}
\begin{itemize}
\item Deadlock is a condition where you try to lock a mutex, but for some reason
the mutex isn't able to lock.
\item One thread tries to lock the same mutex twice.
\item You have two threads and two mutexes $A$ and $B$. When thread 1 locks
mutex $A$ successfully, it tries to lock mutex $B$. However, right before mutex
$B$ gets locked, thread 2 locks mutex $B$ and tries to lock mutex $A$.
\item A way to avoid this is to make sure all threads lock mutexes in the same
order.
\item Order must be preserved otherwise there will be a race condition.
\end{itemize}

\subsection{Reader-Writer Lock}
\begin{itemize}
\item Very useful because a large number of applications fall into this
character.
\item If you have some piece of data, such as database code that is multi-
threaded.
\item We have a file that represents the data.
\item If we update a record, we need to write to this file.
\item If we read from the file, we do not need to write to this file. We are
just accessing it.
\item Everyone can read at the same time because nothing it getting updated.
\item If we use a mutex, we will have a build up in time taken to read the file
if everyone is just reading.
\item Using a \code{rwlock} allows you to lock the file in read or write mode.
\item Write lock is exclusive. You lock in write mode and no other processes can
proceed.
\item If you grab the lock in read mode, it's ok for other threads to call read
lock and lock it at the same time.
\item If everyone is locking in \code{rdlock}, then there is no locking for
processes that are just reading.
\item If one thread grabs a read lock and before it unlocks the process, a write
lock is called. The write lock will not be allowed in as the writer must wait
until all reading processes are finished.
\item If there is a continuous stream of read locks, then no write locks will
not be able to enter into the file.
\item One method to avoid this problem is when the writer is called, block all
new read locks.
\item While a writer is pending, the subsequent readers would have to wait until
the writer lock is done.
\item Incrementing and decrementing a number is a non-atomic method, so we need
to surround it with a lock.
\end{itemize}

\subsection{Condition Variables}
\begin{itemize}
\item You have this variable that the thread can wait on.
\item The variable status changes when the thread calls signal or broadcast.
\item Suppose you have a queue of data items, this queue is in shared memory.
\item When a thread makes a condition happen, it will notify all other threads
that this condition has occurred.
\end{itemize}

\end{document}
