\documentclass[]{article}
\usepackage[all]{xy}
\usepackage{amsmath}
\usepackage{amssymb}
\usepackage{amsthm}
\usepackage{enumitem}
\usepackage{indentfirst}
\usepackage{listings}
\usepackage{multirow}
\usepackage{tikz}
\usepackage{tikz-qtree}
\usepackage{tipa}
\begin{document}

\newcommand{\code}{\texttt}
\newtheorem{thm}{Theorem}
\title{Operating Systems \\ COMS W4118 \\ Lecture 10}
\author{Alexander Roth}
\date{2015 -- 02 -- 24}
\maketitle

Jae likes to go other threads and stuff that people should have read in the
textbook...
\begin{itemize}
\item Be sure to notify when you are in the lock.
\end{itemize}

\section{Nonblocking I/O and I/O Multiplexing}
\subsection{Nonblocking I/O}
\begin{itemize}
\item Certain I/O operations are defined as slow.
\item Slow operations are defined as operations that can block forever.
\item A slow call can be nonblocking (returns immediately if nothing is there).
\item When you set a file descriptor to nonblocking, it will only be able to
write what it can write immediately and it will fail otherwise.
\end{itemize}

\subsection{I/O Multiplexing}
\begin{itemize}
\item \code{select} allows you to find which file descriptors you can use that
aren't blocking.
\item \code{select} will block you until you are ready for \code{reads} and
\code{writes}.
\item \code{maxfdp1} is the file descriptor with the largest number plus 1. The
select function will check a series of bytes up to that point for ready file
descriptors.
\end{itemize}

\end{document}
