\documentclass[]{article}
\usepackage[all]{xy}
\usepackage{amsmath}
\usepackage{amssymb}
\usepackage{amsthm}
\usepackage{enumitem}
\usepackage{indentfirst}
\usepackage{listings}
\usepackage{multirow}
\usepackage{tikz}
\usepackage{tikz-qtree}
\usepackage{tipa}
\begin{document}

\newcommand{\code}{\texttt}
\newtheorem{thm}{Theorem}
\title{Operating Systems \\ COMS W4118 \\ Lecture 6}
\author{Alexander Roth}
\date{2015 -- 02 -- 10}
\maketitle

\section{General Overiew}
\begin{itemize}
\item Mechanisms for two different processes to communicate (Interprocess
communication).
\end{itemize}

\section{File Sharing}
\begin{itemize}
\item You can have two file descriptors share a file by calling \code{dup(1)}.
\item When a parent process forks and create a child process, all the file
descriptors from the parent process get \code{dup}ed into the child's file
descriptor table.
\end{itemize}

\section{Pipes}
\begin{itemize}
\item \code{pipe} is a system call and you pass an array of two integers to it.
\item Arguments are \code{int *fd}.
\item When called, the kernel will create a data structure called a \code{pipe},
that will connect to the two file descriptors together.
\item \code{fd\lbrack 1 \rbrack} will be the writing end of the pipe.
\item \code{fd\lbrack 0 \rbrack} will be the reading end of the pipe.
\item The file descriptor numbers will not be 0 or 1. They will be whatever was
available next.
\item After you call \code{pipe}, you immediately \code{fork} to copy the
descriptors into a child process.
\item This is a half-duplex connection.
\item Whichever file descriptor reads first, then the data will be removed from
the pipe.
\item You must determine what kind of commuincation you desire.
\item \code{STDIN\_FILENO} is 1.
\item \code{dup(2)} copies a file descriptor into another given file descriptor.
\item When useing \code{dup(2)} check that the descriptor you want to copy over
another value, you will want to check that the descriptor is not the same
number.
\item \code{NULL} is sent when there is only one file descriptor open.
\item Write a program that is modeled like the example, but takes another
program name, have it run, and have the output fed into the pager.
\item \code{pipe} is the oldest and simplest interprocess method; however, it is
one way and only related processes can use it.
\end{itemize}

\section{Other Interprocess Systems}
\begin{itemize}
\item There are three other ways to control interprocess communication.
\begin{enumerate}
\item Message queues
\item Semaphores
\item Shared memory
\end{enumerate}
\item All are terribly implemented.
\item Hard to clean-up because there is no reference counting.
\item POSIX refers to the modern Linux standard.
\end{itemize}

\subsection{Shared Memory}
\begin{itemize}
\item Gives you the ability to share a piece of memory (like an array) between
two processes.
\end{itemize}

\subsection{Message Queue}
\begin{itemize}
\item Like Socket, handles communications between processes.
\item Use single-domain-host sockets instead of message queues as they are
easier to use.
\end{itemize}

\subsection{Semaphores}
\begin{itemize}
\item Synchronize actions across processes
\end{itemize}

\section{Memory Mapping}
\begin{itemize}
\item When you compile a program into an executable, it uses a shared object
that controls the standard library.
\item A static link is a \code{.o} file.
\item A dynamic link holds a link to some larger library to load the code in at
run time.
\item The advantage is that you have one copy of standard library on your
system. and all programs will load the necessary parts at run time.
\item This is done by memory-mapping.
\item Used for writing and reading to file and also for executing code.
\item You call \code{mmap} when you want the starting address of a mapped
region.
\begin{description}
\item[addr] Usually pass \code{NULL}, the specific area of the memory.
\item[len] Length of the memory
\item[fd] File descriptor on where to map.
\item[offset] The offset of the file.
\end{description}
\item If you want to zero out a file, you read from \code{/dev/zero}.
\item You use \code{mmap} to create a shared reason by mapping \code{/dev/zero/}
to it to zero out the region.
\item \code{mmap} can be used for unrelated tasks as long as there is a file underneath.
\end{itemize}

\end{document}
