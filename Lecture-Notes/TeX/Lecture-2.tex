\documentclass[]{article}
\usepackage[all]{xy}
\usepackage{amsmath}
\usepackage{amssymb}
\usepackage{amsthm}
\usepackage{enumitem}
\usepackage{indentfirst}
\usepackage{listings}
\usepackage{multirow}
\usepackage{tikz}
\usepackage{tikz-qtree}
\usepackage{tipa}
\begin{document}

\newtheorem{thm}{Theorem}
\title{Operating System \\ COMS W4118 \\ Lecture 2}
\author{Alexander Roth}
\date{2015 -- 01 -- 22}
\maketitle

\section{High-level Overview}
\subsection{Four Componenets of a Computer System}
\begin{itemize}
\item Layer system in computer architecture.
\item User programs do not need to know about the low level hardware systems of
a host.
\item The operating system will provide an interface from the hardware to the
user programs.
\item Provides an API for user programs, so they don't need to know about
hardware.
\end{itemize}

\subsection{Operating System Definition}
\begin{itemize}
\item OS manages all resources.
\item Controls execution of programs to prevent errors and improper use of the
computer.
\item Decides between conflicting requests for efficient and fair resource use.
\item The operating system has to juggle tasks.
\item Operating system is a master, it controls and dictates.
\item It is also a server because it takes commands from user programs and does
things to the hardware according to the user program.
\item No universally accepted definition.
\item "Everything a vendor ships when you order an operating system" is a good
approximation.
\item "The one program running at all times on the computer" is the
\textbf{kernel}.
\item Everything else is either a system program or an application program.
\item We will focus on the behavior of the kernel.
\item The kernel is just one particular program that gets run first.
\item The kernel runs as long as the computer is active.
\item The kernel is a block of code that is sitting in memory that will only
operate when the user applications request it to do something.
\item The kernel goes through states of inactivity and activity.
\end{itemize}

\subsection{Computer System Organization}
\begin{itemize}
\item Computer-system operation
\begin{itemize}
\item One or more CPUs, device controllers connect through common bus providing
access to shared memory
\item Concurrent execution of CPU's and devices competing for memory cycles.
\end{itemize}
\item Registers are very small pieces of memory that are embedded into CPU's or
pseudo-CPU's as typically a few bytes. Very fast memory embedded in processors.
\item \textbf{Simplified:} CPU can only do operations on registers (not entirely
true today).
\item Each subsystem in the machine now has their own processing unit, sometimes
even more powerful than the actual CPU.
\end{itemize}

\subsection{Common Functions of Interrupts}
\begin{itemize}
\item \textbf{von Neumann Architecture:} We have a memory that contains both
program and data. The CPU will go through each instruction in memory (encoding
in bytes) and will fetch other memory in order to handle operations.
\item The program counter register handles which instruction is fetched by the
machine.
\item All the CPU does is run the program in memory.
\item The CPU will keep executing instructions, when will it stop?
\item The CPU runs on a timer, and when the timer runs down, control of the
system is transferred to the operating system.
\item Happens hundreds of times a second.
\item When an timer interrupt comes in, the CPU jumps to a piece of known code
within the operating system.
\item This creates the illusion of multiple programs running at the same time,
when in actuality one program is running at once, while the others are waiting
in the background.
\item CPU can only execute one thread of byte-code at a time. Then a time
interrupt occurs.
\item CPU will save the state to somewhere else and execute another program
elsewhere.
\item This is beneficial to multi-user systems, such as Unix.
\item Operating systems have scheduling algorithms that handle which processes
are being executed when and with what kind of priority they are given.
\item The operating system will preempt programs in order to allow another
process to run in the meantime.
\item Interrupt transfers control to the interrupt service routine generally,
through the \textbf{interrupt vector}, which contains the addresses of all the
service routines.
\item Interrupt architecture must save the address of the interrupted
instruction.
\end{itemize}

\subsubsection{Interrupts}
\begin{itemize}
\item Three ways interrupts happen:
\begin{enumerate}
\item Timing interrupt: as described above
\item All the controllers (I/O devices) have the ability to be interrupted by
the disk. Example: The disk interrupt will tell the operating system that the
information being read in from the disk is available. Typically scheduling
algorithms give priority to I/O events as that is what users interact with. Used
to signal completion of tasks or I/O errors.
\item When a program fails, a software interrupt will occur. Error handling or
errors will occur. The programmer will set up memory locations that are invalid,
these areas will through interrupts if they are accessed.
\end{enumerate}
\item Switching from one program to another is very expensive. \textbf{Context
switching}, must save state and data from program and then switch to another
program and bring up the old state there.
\item Exceptions occur from running a piece of code in the kernel that throws an
error.
\item System call mechanisms also piggy-back on interrupt mechanisms.
\item Segmentation fault is a software interrupt.
\end{itemize}

\subsection{Interrupt Timeline}
\begin{itemize}
\item When a program requests an I/O, it will take quite some time for the CPU.
\item You do not want to waste CPU time, but you want to know exactly when it is
done, so we use interrupts.
\end{itemize}

\subsection{I/O Structure}
Two ways to handle I/O:
\begin{enumerate}
\item Call I/O and have the CPU wait. (Old implementation)
\item Operating system will juggle processes so you don't waste CPU.
\end{enumerate}

\subsection{Storage-Device Hierarchy}
\begin{itemize}
\item There's a memory hierarchy that everyone will learn if they take a
database course.
\begin{enumerate}
\item Registers
\item Cache
\item Main Memory
\item Solid-state disk
\item Hard Disk
\item Optical Disk
\item Magnetic Tape
\end{enumerate}
\item Higher up; smaller, faster memory.
\item Lower down; bigger, slower memory.
\item \emph{Vector} is the best memory structure ever because the cache makes
\emph{linked list} slow as fuck.
\item A structure has to be aligned so that it fits into 1024 bytes.
\item Making sure that code is easily accessed and easily cacheable is very
important to the operating system.
\end{itemize}

\subsection{How a Modern Computer Works}
\subsubsection{Old Method}
\begin{itemize}
\item The CPU issues an I/O request.
\item The device it was issued to fulfills the request and sends back an
interrupt.
\item The CPU will transfer the data to a local buffer from the I/O device.
\item Takes a long time.
\end{itemize}

\subsubsection{Modern Method}
\begin{itemize}
\item Direct Memory Access.
\item CPU sends request to device.
\item Device learns where to send the information and where to get it.
\item Device doesn't need to interrupt the CPU.
\end{itemize}

\subsection{Transition from User to Kernel Mode}
\begin{itemize}
\item Timer to prevent infinite loop/process hogging resources
\begin{itemize}
\item Timer is set to interrupt the computer after some time period.
\item Keep a counter that is decremented by the physical clock.
\item Operating system set the counter (privileged instruction)
\item When counter zero generates an interrupt
\item Set up before scheduling process to regain control or terminate program
that exceeds allotted time.
\end{itemize}
\item Dual modes of operation. Privileged operations can only occur when the CPU
is in privileged mode (kernel mode).
\item What is available to the user are C functions that act as system calls.
\item Library code activates these smaller C functions.
\item Bottom-level C functions are just wrappers around the assembly code.
\end{itemize}

\end{document}
