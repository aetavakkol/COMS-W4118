\documentclass[]{article}
\usepackage[all]{xy}
\usepackage{amsmath}
\usepackage{amssymb}
\usepackage{amsthm}
\usepackage{enumitem}
\usepackage{indentfirst}
\usepackage{listings}
\usepackage{multirow}
\usepackage{tikz}
\usepackage{tikz-qtree}
\usepackage{tipa}
\newcommand{\code}{\texttt}
\begin{document}

\newtheorem{thm}{Theorem}
\title{Operating Systems \\ COMS W4118 \\ Extras 1}
\author{Alexander Roth}
\date{2015 -- 01 -- 29}
\maketitle

\section*{Extra Questions}
\begin{enumerate}
\item What is partitioning?

Partitioning a hard drive allows one to logically divide the available space
into sections that can be accessed independently of one another. It is the act
of dividing a hard disk drive into multiple logical storage units refered to as
\emph{partitions}, to treat one physical disk drive as if it were multiple
disks, so that a different file system can be used on each partition.

\item What is a file system?

A \emph{file system} is a means to organize data expected to be retained after a
program terminates by providing procedures to store, retrieve and update data,
as well as manage the available space on the device(s) which contain it. A file
system organizes data in an efficient manner and is tuned to the specific
characteristics of the device.

\item We chose ext4 file system. What other file systems are there?

There are a number of different file systems. According to the Arch Linux wiki,
the file systems are
\begin{description}
\item[Btrfs] ``Better FS'', is a \textbf{new filesystem with powerful features
similar to Sun/Oracle's excellent ZFS}.
\item[exFAT] \textbf{Microsoft file system optimized for flash drive}
\item[ext2] \textbf{Second Extended Filesystem} is an established, mature
GNU/Linux filesystem that is very stable.
\item[ext3] \textbf{Third Extended Filesystem} is essentially the ext2 system
with journaling support and write barriers.
\item[ext4] \textbf{Fourth Extended Filesystem} is a newer filesystem that is
also compatible with ext2 and ext3. It provides support for volumes with sizes
up to 1 exabyte and files sizes up to 16 terabytes.
\item[F2FS] \textbf{Flash-Friendly File System} is a flash file system created
by Kim Jaegeuk at Samsung for the Linux operating system kernel.
\item[JFS] IBM's \textbf{Journaled File System} was the first filesystem to
offer journaling.
\item[NILFS2] \textbf{New Implementation of a Log-structure File System} was
developed by NTT. It records all data in a continuous log-like format that is
only append to and never overwritten.
\item[NTFS] \textbf{File system used by Windows.} NTFS has several technical
improvements over FAT and HPFS (High Performance File System), such as improved
support for metadata, and the use of advanced data structure to improve
performance, reliability, and disk space utilization, plus additional
extensions, such as security access control lists and file system journaling.
\item[Reiser4] \textbf{Successor to the ReiserFS file system}, developed from
scratch by Namesys and sponsored by DARPA as well as Linspire, it uses B*-trees
in conjunction with the dancing tree balancing approach, in which underpopulated
nodes will not be merged until a flush to disk except under memory pressure or
when a transaction completes.
\item[ReiserFS] \textbf{Hans Reiser's high-performance journaling FS (v3)} uses
a very interesting method of data throughput based on an unconventional and
creative algorithm.
\item[VFAT] \textbf{Virtual File Allocation Table} is technically simple and
supported by virtually all existing operating systems.
\item[XFS] \textbf{Early journaling filesystem originally developed by Silicon
Graphics} for the IRIX operating system and ported to GNU/Linux. It provides
very fast throughput on large files and filesystems and is very fast at
formatting and mounting.
\item[ZFS] \textbf{Combined file system and logical volume manager designed by
Sun Microsystems}
\end{description}

\item What is a swap partition or a swap file?

Linux divides its physical RAM into chunks of memory called pages. Swapping is
the process whereby a page of memory is copied to the preconfigured space on the
hard disk, called swap space, to free up that page of memory. The combined sizes
of the physical memory and the swap space is the amount of virtual memory
available.

\item What's the difference between the two files?

The swap partition is an independent section of the hard disk used solely for
swapping; no other files can reside there. The swap file is a special file in the
filesystem that resides amongst your system and data files.

\item What does mounting a file system mean?

The \code{mount} command instructs the operating system that a file system is
ready for use, and associates it with a particular point in the overall file
system hierarchy and sets options relating to its access. Mounting makes file
systems, files, directories, devices, and special files available to the user.

\item What does \code{/etc/fstab} contain?

The \code{/etc/fstab} file can be used to define how disk partitions, various
other block devices, or remote filesystems should be mounted into the
filesystem.

Each filesystem is described in a separate line. These definitions will be
converted into \code{systemd} mount units dynamically at boot, and when the
configuration of the system manger is reloaded.

\item What is \code{chroot}?

\code{Chroot} is an operation that changes the apparent root directory for the
current running process and their children. A program that is run in such a
modified environment cannot access files and commands outside that environmental
directory tree. This modified environment is called a \emph{chroot jail}.
Changing root is commonly done for performing system maintenance on systems
where booting and/or logging in is no longer possible.

\item Why did the installation process have \code{chroot}?

The installation process has \code{chroot} because we were modifying the system
without a root user or without any other users in the system. Thus, we needed a
way to access the system directory without creating a root user. The system
still did not have a bootloader installed.

\item Did you reboot before or after \code{chroot}?

Rebooted after the \code{chroot} once the bootloader was installed.

\item Can you describe how the installation works at a very high level in a few
sentences?

Yes, first we set up an internet connection so we can download from our mirror
sites during the installation process. Then, we prepare a storage device by
partition the drive and attach our filesystem (ext4) to it. We then mount the
disk image file system from the arch linux image onto our virtual machine. Next,
we download the necessary packages from the base sites and customize the kernel
with the necessary regional information. Finally, we download the bootloader and
build it. We then reboot the system and eject the disk image.

\item What does \code{systemctl enable dhcpcd.service} do?

\code{systemctl} is the command run to control \code{systemd} (the service
manager for Linux). By calling \code{systemctl enable dhcpcd.service}, we are
enabling the dhcpcd.service unit at bootup of the virtual machine.

\item Does it have any immediate effects?

No, it does not.

\item Does it set something up for the next reboot?

Yes, the service will be enabled on reboot. More specifically the dhcpcd service
will be enabled after reboot.

\item What is \code{sudo}?

\code{sudo} (``substitute user do'') allows a system administrator to delegate
authority to give certain users (or groups of users) the ability to run some (or
all) commands as root or another user while providing an audit trail of the
commands and their arguments.

\item How does \code{sudo} work?:

\code{sudo} enables root privileges only when needed as per the \code{setuid}
command. It sets the effective user ID of the calling process.

\item How does the Arch Linux packaging system work?
\item When you install software through pacman, do you get the source code too?
If not, how do you go about getting it?
\item What is an X server?
\item Can you describe the client-server architecture of the X window system?
\item What does startx or startxfce4 do?
\item What is a kernel module?
\item What kernel modules are you running?
\end{enumerate}

\end{document}
