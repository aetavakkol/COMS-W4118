\documentclass[]{article}
\usepackage[all]{xy}
\usepackage{amsmath}
\usepackage{amssymb}
\usepackage{amsthm}
\usepackage{enumitem}
\usepackage{indentfirst}
\usepackage{listings}
\usepackage{multirow}
\usepackage{tikz}
\usepackage{tikz-qtree}
\usepackage{tipa}
\newcommand{\code}{\texttt}
\begin{document}

\newtheorem{thm}{Theorem}
\title{Operating Systems \\ COMS W4118 \\ Reading Notes 3}
\author{Alexander Roth}
\date{2015 -- 02 -- 04}
\maketitle

\section*{Advanced Programming in the UNIX Environment \\
Chapter 7: Process Environment}
\subsection*{7.6 Memory Lay out of a C Program}
\begin{itemize}
\item A C program has been composed of the following pieces:
\begin{itemize}
\item Text segment, consisting of the machine instructions that the CPU
executes.
\item Initialized data segment, usually called simply the data segment,
containing variables that are specifically initialized in the program.
\item Uninitialized data segment, often called the ``bss'' segment, named after
an ancient assembler operator that stood for ``block started by symbol.''
\item Stack, where automatic variables are stored, along with information that
is saved each time a function is called.
\item Heap, where dynamic memory allocation usually takes place.
\end{itemize}
\item The only portions of the program that need to be saved in the program file
are the text segment and the initialized data.
\end{itemize}

\subsection*{7.7 Shared Libraries}
\begin{itemize}
\item Shared libraries remove the common library routines from the executable
file, instead maintaining a single copy of the library routine somewhere in
memory that all processes reference.
\item Shared libraries reduce the size of each executable file, but add some
runtime overhead.
\item Library functions can be replaced with new verions without having to
relink edit every program.
\end{itemize}

\subsection*{7.8 Memory Allocation}
\begin{itemize}
\item There are three functions for memory allocation:
\begin{itemize}
\item \code{malloc}, which allocates a specified number of bytes of memory.
\item \code{calloc}, which allocates space for a specified number of objects of
a specified size.
\item \code{realloc}, which increases or decreases the size of a previously
allocated area.
\end{itemize}
\end{itemize}

\code{\#include <stdlib.h>}

\code{void *malloc(size\_t \emph{size});}

\code{void *calloc(size\_t \emph{nobj}, size\_t \emph{size});}

\code{void *realloc(void *\emph{ptr}, size\_t \emph{newsize});}

All three return: non-null pointer if OK, \code{NULL} on error \\

\code{void free(void *\emph{ptr});}

\begin{itemize}
\item The pointer returned by the three allocation functions is guaranteed to be
suitably aligned so that it can be used for any data object.
\item The function \code{free} causes the space pointed to by \emph{ptr} to be
deallocated.
\item The allocation routeins are usually implemented with the \code{sbrk(2)}
system call. This system call expands (or contracts) the heap of the process.
\item Most verions of \code{malloc} and \code{free} never decrease their memory
size. Freed space is kept in the \code{malloc} pool.
\item Most implementations allocate more space than requested and use the
additional space for record keeping -- the size of the block, a pointer to the
next allocated block, and the like.
\end{itemize}

\section*{Advanced Programming in the UNIX Environment \\
Chapter 14: Advacned I/O}
\subsection*{14.8 Memory-Mapped I/O}
\begin{itemize}
\item Memory-mapped I/O lets us map a file on disk into a buffer in memory so
that, when we fetch bytes from the buffer, the corresponding bytes of the file
are read.
\end{itemize}

\code{\#include <sys/mman.h>}

\code{void *mmap(void *\emph{addr}, size\_t \emph{len}, int \emph{prot}, int
\emph{flag}, int \emph{fd}, off\_t \emph{off});}

Returns: starting address of mapped region if OK, \code{MAP\_FAILED} on error.

\begin{itemize}
\item The \emph{addr} arguments lets us specify the address where we want the
mapped region to start.
\item The \emph{fd} argument is the file descriptor specifying the file that is
to be mapped.
\item The \emph{len} argument  is the number of bytes to map.
\item The \emph{off} is the starting offset in the file of the bytes to map.
\item The \emph{prot} argument specifies the protection of the mapped region.
\item Each implementation has additional \code{MAP\_xxx} flag values, which are
specific to that implementation.
\item The \code{SIGSEGV} signla is normally used to indicate that we tried to
access memory that is not available to us, such as read-only memory.
\item The \code{SIGBUS} signal can be generated if we access a portion of the
mapped region that does not make sense at the time of the access.
\itme A memory-mapped region is inherited by a chile across a \code{fork} but it
is not inherited by the new program across an \code{exec}.
\end{itemize}


\section*{Advanced Programming in the UNIX Environment \\
Chapter 15: Interprocess Communication}
\subsection*{15.2 Pipes}
\begin{itemize}
\item Pipes have two limitations:
\begin{enumerate}
\item Historically, they have been half duplex
\item Pipes can be used only between processes that have a common ancestor.
\end{enumerate}
\item The shell creates a separate process for each command and links the
standard output of one process to the standard input of the next using a pipe.
\end{itemize}

\code{\#include <unistd.h>}

\code{int pipe(int \emph{fd\lbrack 2 \rbrack});}

Returns: 0 if OK, -1 on error

\begin{itemize}
\item Two file descriptors are returned through the \emph{fd} argument:
\begin{description}
\item[fd\lbrack 0 \rbrack] opened for reading
\item[fd\lbrack 1 \rbrack] opened for writing.
\end{description}
\item  When one end of a pipe is closed, two rules apply.
\begin{enumerate}
\item If we \code{read} from a pipe whose write end has been closed, \code{read}
returns 0 to indicate an end of file after all teh data has been read.
\item If we \code{write} to a pipe whose read end has been closed, the signal
\code{SIGPIPE} is generated.
\end{enumerate}
\item When we're writing to a pipe, the constant \code{PIPE\_BUF} specifies the
kernel's pipe buffer size.
\end{itemize}

\subsection*{15.10 POSIX Semaphores}
\begin{itemize}
\item Semaphores are useful tool in the prevention of race conditions.
\item Semaphores are records of how many units of a particular resource are
available, coupled with operations to \emph{safely} adjust that record as units
are required or become free, and, if necessary, wait until a unit of the
resource becomes available.
\end{itemize}

\code{\#include <semaphore.h>}

\code{sem\_t *sem\_open(const char *\emph{name}, int \emph{oflag}, \ldots\, /*
mode\_t \emph{mode}, unsigned int \emph{value} */);}

Returns: Pointer to semaphore if OK, \code{SEM\_FAILED} on error

\begin{itemize}
\item When useing an existing named semaphore, we specify only two arguemnts:
the name of the semaphore and a zero value for the \emph{oflag} argument.
\item When we specify the \code{O\_CREAT} flag, we need to provide two
additional arguments. The \emph{mode} argument specifies who can access the
semaphore. The \emph{value} arguemnt is used to specify the initial value for
the semaphore when we create it.
\item To promote portability, we must follow certain conventions when selecting
a semaphore name.
\begin{itemize}
\item The first character in the name should be a slash (\code{/}).
\item The name should contain no other slashes to avoid implementation-defined
behavior.
\item The maximum length of the semaphore name is implementation defined.
\end{itemize}
\item The \code{sem\_open} function returns a semaphore pointer that we can pass
to other semaphore functions when we operate on the semaphore.
\end{itemize}

\code{\#include <semaphore.h>}

\code{int sem\_close(sem\_t *\emph{sem});}

Returns: 0 if OK, -1 on error

\begin{itemize}
\item If our process exits without having first called \code{sem\_close}, the
kernel will close any open semaphores automatically.
\end{itemize}

\code{\#include <semaphore.h>}

\code{int sem\_unlink(const char *\emph{name});}

Returns: 0 if OK, -1 on error

\begin{itemize}
\item The \code{sem\_unlink} function removes the name of the semaphore. If
there are no open references to the semaphore, then it is destroyed.
\end{itemize}

\code{\#include <semaphore.h>}

\code{int sem\_trywait(sem\_t *\emph{sem});}

\code{int sem\_wait(sem\_t *\emph{sem});}

Both return: 0 if OK, -1 on error

\begin{itemize}
\item The \code{sem\_wait} function will block if the semaphore count is 0.
\end{itemize}

\code{\#include <semaphore.h>}
\code{\#include <time.h>}

\code{int sem\_timedwait(sem\_t *restrict \emph{sem}, const struct timespec
*restrict \emph{tsptr});}

Returns: 0 if OK, -1 on error

\begin{itemize}
\item The \emph{tsptr} arugment specifies the absolute time when we want to give
up waiting for the semaphore.
\end{itemize}

\end{document}
