\documentclass[]{article}
\usepackage[all]{xy}
\usepackage{amsmath}
\usepackage{amssymb}
\usepackage{amsthm}
\usepackage{enumitem}
\usepackage{indentfirst}
\usepackage{listings}
\usepackage{multirow}
\usepackage{tikz}
\usepackage{tikz-qtree}
\usepackage{tipa}
\begin{document}

\newcommand{\code}{\texttt}
\newtheorem{thm}{Theorem}
\title{Operating Systems \\ COMS W4118 \\ Reading Notes 6}
\author{Alexander Roth}
\date{2015 -- 02 -- 23}
\maketitle

\section*{Advanced Programming in the Unix Environment \\
Chapter 14: Advanced I/O}

\subsection*{14.2 Nonblocking I/O}
\begin{itemize}
\item ``Slow'' system calls are those that can block forever.
\item Slow systems calls are
\begin{itemize}
\item Reads that can block the caller forever if data isn't present with certain
file types (pipes, terminal devices, and network devices).
\item Writes that can block the caller forever if the data can't be accepted
immediately by these same file types (e.g. no room in the pipe, network flow
control)
\item Opens that block until some condition occurs on certain file types (such
as an open of a terminal device that waits until an attached modem answers the
phone, or an open of a FIFO for writing only, when no other process has the FIFO
open for reading)
\item Reads and writes of files that have mandatory record locking enabled.
\item Certain \code{ioctl} operations
\item Some of the interprocess communication functions.
\end{itemize}
\item System calls related to disk I/O are not considered slow, even though they
can block the caller temporarily.
\item Nonblocking I/O lets us issue an I/O operation and not have it block
forever.
\item If the operation cannot be completed, teh call returns immediately with an
error noting that the operation would have blocked.
\item There are two ways to specify nonblocking I/O for a given descriptor.
\begin{enumerate}
\item If we call \code{open} to get the descriptor, we can specify the
\code{O\_NONBLOCK} flag
\item For a descriptor that is already open, we call \code{fcntl} to turn the
\code{O\_NONBLCOK} file status flag.
\end{enumerate}
\end{itemize}

\subsection*{14.4 I/O Multiplexing}
\begin{itemize}
\item \emph{Asynchronous I/O} is the technique by which we tell the kernel to
notify us with a signal when a descriptor is ready for I/O.
\item It is not portable and we cannot tell which descriptor is ready.
\item \emph{I/O multiplexing} is when we build a list of descriptors that we are
interested in and call a function that doesn't return until one of the
descriptors is ready for I/O.
\item \code{poll}, \code{pselect}, and \code{select} allow us to perform I/O
multiplexing.
\item On return from these functions, we are told which descriptors are ready
for I/O.
\end{itemize}

\subsubsection*{14.4.1 \code{select} and \code{pselect} Functions}
\begin{itemize}
\item The \code{select} function is portable to all POSIX-compatible platforms.
\item The arguments we pass to \code{select} tell the kernel
\begin{itemize}
\item Which descriptors we're interested in.
\item Which conditions we're interested in for each descriptor
\item How long we want to wait
\end{itemize}
\item On the return from select, the kernel tell us
\begin{itemize}
\item The total count of the number of descriptors that are ready
\item Which descriptors are ready for each of the three conditions (read, write,
or exception condition)
\end{itemize}
\end{itemize}

\code{\#include <sys/select.h>}

\code{int select(int \emph{maxfdp1}, fd\_set *restrict \emph{readfds}, fd\_set
*restrict \emph{writefds}, fd\_set *restrict \emph{exceptfd}, struct timeval
*restrict \emph{tvptr});}

Returns: count of ready descriptors, 0 on timeout, -1 on error

\begin{itemize}
\item The last argument specifies how long we want to wait in terms of seconds
and microseconds.
\begin{description}
\item[\code{tvptr == NULL}] Wait forever. Can be interrupted if we catch a
signal. Return is made when one of the specified descriptors is ready or when a
signal is caught. If a signal is caught, returns -1 with \code{errno} set to
\code{EINTR}.
\item[\code{tvptr->tv\_sec == 0 \&\& tvptr->tv\_usec == 0}] Don't wait at all.
All specified descriptors are tested, and return is made immediately. Polls the
system to find out the status of descriptors without blocking in the function.
\item[\code{tvptr->tv\_sec ~= 0 || tvptr->tv\_usec != 0}] Wait the specified
number of seconds and microseconds. Return is made when one of the specified
descriptors is ready or when the timeout value expires.
\end{description}
\item The middle three arguments are pointers to \emph{descriptor sets}.
\item A descriptor set is stored in an \code{fd\_set} data type.
\item The only thing we can do with the \code{fd\_set} data type is allocate a
variable of this type, assign a variable of this type to another variable of the
same type, or use one of the following four functions on a variable of this
type.
\end{itemize}

\code{\#include <sys/select.h>}

\code{int FD\_ISSET(int \emph{fd}, fd\_set *\emph{fdset});}

Returns: nonzero if \emph{fd} is in set, 0 otherwise \\

\code{void FD\_CLR(int \emph{fd}, fd\_set *\emph{fdset});}

\code{void FD\_SET(int \emph{fd}, fd\_set *\emph{fdset});}

\code{void FD\_ZERO(fd\_set *\emph{fdset});}

\begin{itemize}
\item These interface can be implemented as either macros or functions.
\item The first argument of \code{select} is really a count of the number of
descriptor to check.
\item There are three possible return values from \code{select}.
\begin{enumerate}
\item A return value of -1 means that an error occurred.
\item A return value of 0 means that no descriptors are ready.
\item A positive return value specifies the number of descriptors that are
ready.
\end{enumerate}
\item A descriptor is said to be ready when
\begin{itemize}
\item f it is in the read set and if a \code{read} from that descriptor won't
block.
\item if it is in the write set and if a \code{write} from that descriptor won't
block.
\item If it is in the exception set if an exception condition is pending on that
descriptor.
\item File descriptors for regular files always return ready for reading,
writing, and exception conditions.
\end{itemize}
\item A descriptor blocking does not affect if \code{select} blocks.
\end{itemize}

\end{document}
