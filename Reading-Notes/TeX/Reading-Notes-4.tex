\documentclass[]{article}
\usepackage[all]{xy}
\usepackage{amsmath}
\usepackage{amssymb}
\usepackage{amsthm}
\usepackage{enumitem}
\usepackage{indentfirst}
\usepackage{listings}
\usepackage{multirow}
\usepackage{tikz}
\usepackage{tikz-qtree}
\usepackage{tipa}
\begin{document}

\newcommand{\code}{\texttt}
\newtheorem{thm}{Theorem}
\title{Operating Systems \\ COMS W4118 \\ Reading Notes 4}
\author{Alexander Roth}
\date{2015 -- 02 -- 19}
\maketitle

\section*{Advanced Programming in the Unix Environment \\
Chapter 9: Process Relationships}
\subsection*{9.4 Process Groups}
\begin{itemize}
\item A process group is a collection of one or more processes, usually
associated with the same job, that can receive signals from the same terminal.
\item Each process group has a unique process group ID.
\item The function \code{getpgrp} returns the process group ID of the calling
process.
\end{itemize}

\code{\#include <unistd.h>}

\code{pid\_t getpgrp(void);}

Returns: process group ID of calling process.\\

\code{\#include <unistd.h>}

\code{pid\_t getpgid(pid\_t \emph{pid});}

Returns: process group ID if OK, -1 on error.

\begin{itemize}
\item Each process group can have a process group leader. The leader is
identified by its process group ID being equal to its process ID.
\item The process group lifetime is the period of time that begins when a group
is created and ends when the last remaining process leaves the group.
\item A process joins an existing process group or creates a new process group
by calling \code{setpgid}.
\end{itemize}

\code{\#include <unistd.h>}

\code{int setpgid(pid\_t \emph{pid}, pid\_t \emph{pgid});}

Returns: 0 if OK, -1 on error.

\begin{itemize}
\item Sets the process group ID to \emph{pgid} in the process whose process ID
equals \emph{pid}.
\item If the two arguments are equal, the process specified becomes a process
group leader.
\item A process can set the process group ID of only itself or any of its
children.
\item This function is often called after a \code{fork} by the parent process to
set the child's process group ID and by the child to set it's own child group
ID. We have the redundancy to avoid any race conditions.
\end{itemize}

\subsection*{9.5 Sessions}
\begin{itemize}
\item A session is a collection of one or more process groups.
\item The processes in a process group are usually placed into their respective
process group by a shell pipeline.
\item A process establishes a new session by calling the \code{setsid} function.
\end{itemize}

\code{\#include <unistd.h>}

\code{pid\_t setsid(void);}

Returns: process group ID if OK, -1 on error

\begin{itemize}
\item If the calling process is not a process group leader, this function
creates a new session.
\item Three things happen afterwards:
\begin{enumerate}
\item The process becomes the \emph{session leader} and is the only process in
this new session. A session leader is the process that creates a session.
\item The process becomes the process group leader of a new procress group; the
process ID of the calling process is the process group ID.
\item The process has no controlling terminal.
\end{enumerate}
\item This function returns an error if the caller is already a process group
leader.
\item The \code{getsid} function returns the process group ID of a process's
session leader.
\item The session leader is always the leader of a process group.
\end{itemize}

\code{\#include <unistd.h>}

\code{pid\_t getsid(pid\_t \emph{pid});}

Returns: session leader's process group ID if OK, -1 on error

\subsection*{9.6 Controlling Terminal}
\begin{itemize}
\item A session can have a single \emph{controlling terminal}, which is usually
the terminal device or pseudo terminal device on which we log in.
\item The session leader that establishes the connection to the controlling
terminal is called the \emph{controlling process}.
\item The proces groups within a session can be divided into a single
\emph{foreground process group} and one or more \emph{background process
groups}.
\item If a session has a controlling terminal, it has a single foreground proces
group and all other process groups in the session are background process groups.
\item Whenever we press the terminal's interrupt key, the interrupt signal is
sent to all processes in the foreground process group.
\item Whenever we press the terminal's quit key, the quit signal is sent to all
processes in the foreground process group.
\item If a modem disconnect is detected by the terminal interface, the hang-up
signal is sent to the controlling process (session leader).
\item A program guarantees it is talking ot the controlling terminal when it
opens the file \code{/dev/tty}. This special file is synonomous within the
kernel for the controlling terminal.
\end{itemize}

\section*{Chapter 10: Signals}
\subsection*{10.8 Reliable-Signal Terminology and Semantics}
\begin{itemize}
\item A signal is \emph{generate} for a process when the event that causes the
signal occurs. The event could be a hardware exception, a software condition, or
anything else.
\item A signal is \emph{delivered} to a process when the action for a signal is
taken.
\item During the time between the generation of a signal and its delivery, the
signal is said to be \emph{pending}.
\item A process has the option of \emph{blocking} the delivery of a signal.
\item If a signal that is blocked is generated for a process, and if the action
for that signal is either the default action or to catch the signal, then the
signal remains pending for the process until the process either unblocks the
signal or changes the action to ignore the signal.
\item If the system delivers the signal more than once, we say that the signals
are queued.
\item Most UNIX systems do not queue signals. The kernel simply delivers the
signal once.
\item POSIX.1 does not specify the order in which the signals are delivered to
the process. Signals related to the current state of the process should be
delivered before other signals.
\item Each process has a \emph{signal mask} that defines the set of signals
currently blocked from delivery to that process.
\end{itemize}

\subsection{10.9 \code{kill} and \code{raise} Functions}
\begin{itemize}
\item The \code{kill} function sends a signal to a process or a group of
processes.
\item The \code{raise} function allows a process to send a signal to itself.
\end{itemize}

\code{\#include <signal.h>}

\code{int kill(pid\_t \emph{pid}, int \emph{signo});}

\code{int raise(int \emph{signo});}

Both return: 0 if OK, -1 on error

\begin{itemize}
\item There are four different conditions for the \emph{pid} argument to
\code{kill}.
\begin{description}
\item[$pid > 0$] The signal is sent to the process whose process ID is
\emph{pid}.
\item[$pid == 0$] The signal is sent to all processes whose process group ID
equals the process group ID of the sender and for which the sender has
permission to send the signal.
\item[$pid < 0$] The signal is sent to all processes whose process group ID
equals the absolute value of \emph{pid} and for which the sender has permission
to send the signal.
\item[$pid == -1$] The signal is sent to all processes on the system for which
the sender has permission to send the signal.
\end{description}
\item A process needs permission to send a signal to another process.
\item The superuser can send a signal to any process.
\item POSIX.1 defines signal number 0 as the null signal.
\item If the \emph{signo} argument is 0, then the normal error checking is
performed by \code{kill}, but no signal is sent.
\item The call to \code{kill} is not atomic.
\end{itemize}

\subsection*{10.10 \code{alarm} and \code{pause} Functions}
\begin{itemize}
\item The \code{alarm} function allows us to set a timer that will expire at a
specified time in the future.
\item When the timer expires, the \code{SIGALRM} signal is generated.
\item The default action with \code{SIGALRM} is to terminate the process.
\end{itemize}

\code{\#include <unistd.h>}

\code{unsigned int alarm(unsigned int \emph{seconds});}

Returns: 0 or number of seconds until previously set alarm

\begin{itemize}
\item The \emph{seconds} value is the number of clock seconds in the future when
the signal should be generated.
\end{itemize}

\end{document}
