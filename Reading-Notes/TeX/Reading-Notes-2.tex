\documentclass[]{article}
\usepackage[all]{xy}
\usepackage{amsmath}
\usepackage{amssymb}
\usepackage{amsthm}
\usepackage{enumitem}
\usepackage{indentfirst}
\usepackage{listings}
\usepackage{multirow}
\usepackage{tikz}
\usepackage{tikz-qtree}
\usepackage{tipa}
\newcommand{\code}{\texttt}
\begin{document}

\newtheorem{thm}{Theorem}
\title{Operating Systems \\ COMS W4118 \\ Reading Notes 2}
\author{Alexander Roth}
\date{2015 -- 01 -- 31}
\maketitle

\section*{Advanced Programming in the Unix Environment \\
Chapter 10: Signals}
\subsection*{10.1 Introduction}
\begin{itemize}
\item Signals are software interrupts.
\item Signals provide a way of handling asynchronous events.
\end{itemize}

\subsection*{10.2 Signal Concepts}
\begin{itemize}
\item Every signal has a name and all names begin with the three characters
\texttt{SIG}.
\item Signal names are all defined by positive integer constants in the header
\texttt{<signal.h>}.
\item No signal has a signal number of 0.
\item It is considered bad form for the kernel to include header files meant
for user-level applications, so information is placed in the kernel header and
then included by the user-level header. Thus, we have \code{<sys/signal.h>}
for Mac OSX.
\item Numerous conditions can generate a signal:
\begin{itemize}
\item Terminal-generated signals occur when users press certain terminal keys.
\item Hardware exceptions generate signals.
\item The \code{kill()} functions allow us to send signals to other processes.
\item Software conditions can generate signals when a process should be notified
by various events.
\end{itemize}
\item The \emph{disposition} or \emph{action} associated with the signal can be
one of three things:
\begin{enumerate}
\item Ignore the signal. (Note: \code{SIGKILL} and \code{SIGSTOP} can never
be ignored.)
\item Catch the signal. (Note: \code{SIGKILL} and \code{SIGSTOP} can never
be caught.)
\item Let the default action apply.
\end{enumerate}
\end{itemize}

\subsection*{10.3 \code{signal} function}
\texttt{\#include <signal.h>}

\texttt{void (*signal(int \emph{signo}, void (*\emph{func})(int)))(int);}

Returns: previous disposition of signal if OK, \code{SIG\_ERR} on error

\begin{itemize}
\item The \emph{signo} argument is just the name of the signal.
\item The value of \emph{func} is
\begin{enumerate}
\item the constant \code{SIG\_IGN}
\item the constant \code{SIG\_DFL}
\item the address of a function to be called when the signal occurs
\end{enumerate}
\item \code{SIG\_IGN} tells the system to ignore the signal.
\item \code{SIG\_DFL} tells the system to follow the default action associated
with the signal.
\item When the address of a function is given, we are arranging a ``catch''
within the system.
\item The function requires two arguments
\begin{enumerate}
\item The first argument, \emph{signo}, is an integer.
\item The second argument is a pointer to a function that takes a single
interger argument and returns nothing.
\end{enumerate}
\end{itemize}

\subsubsection*{Program Start-Up}
\begin{itemize}
\item When a program is executed, the status of all signals is either default or
ignore.
\item We are not able to determine the current disposition of a signal without
changing the disposition of the signal.
\end{itemize}

\subsubsection*{Process Creation}
\begin{itemize}
\item When a process calls \code{fork}, the child inherits the parent's signal
dispositions.
\end{itemize}

\subsection*{10.4 Unreliable Signals}
\begin{itemize}
\item In earlier versions of the UNIX system, signals were unreliable; this
means that signals could get lost: a signal could occur and the process would
never know about it.
\item No way to remember if a signal occured without having to do anything about
it. Could only ignore signal or act on it.
\end{itemize}

\subsection*{10.5 Interrupted System Calls}
\begin{itemize}
\item If a process caught a signal while the process was blocked in a ``slow''
system call, the acll was interrupted.
\item The system returned an error and \code{errno} was set to \code{EINTR}.
\item The slow system calls are those that can block forever.
\begin{itemize}
\item Reads that can block the caller forever if data isn't present with certain
file types.
\item Write that can block the caller forever if the data can't be accepted
immediately by these same file types.
\item Opens on certain file types that block the caller until some condition
occurs.
\item The \code{pause} function and the \code{wait} function.
\item Certain \code{ioctl} operations
\item Some of the interprocess communication functions
\end{itemize}
\item \code{wait} and \code{waitpid} are always interrupted when a signal is
caught.
\end{itemize}

\subsection*{10.6 Reentrant Functions}
\begin{itemize}
\item When a signal that is being caught is handled by a process, the normal
sequence of instructions begin executed by the process is temporarily
interrupted by the signal handler.
\item If the signal handler returns, then the normal sequence of instructions
that the process was executing when the signal was caught continues executing.
\item The Single UNIX Specification specifies the functions that are guaranteed
to be safe to call from within a signal handler.
\item These functions are reetrant and called \emph{async-signal safe}. They
block any signals during operation if delivery of a signal might cause
inconsistencies.
\item As a general rule, when calling fucntions from a signal handler, we should
save and restore \code{errno}.
\item If we call a nonreentrant function from a signal handler, the results are
unpredictable.
\end{itemize}

\section*{Advanced Programming in the Unix Environment \\
Chapter 3: File I/O}
\subsection*{3.1 Introduction}
\begin{itemize}
\item Most file I/O on a UNIX system can be performed using only five functions:
\begin{enumerate}
\item \code{open}
\item \code{read}
\item \code{write}
\item \code{lseek}
\item \code{close}
\end{enumerate}
\item \emph{Unbuffered I/O} means that each \code{read} or \code{write} invokes
a system call in the kernel.
\end{itemize}

\subsection*{3.2 File Descriptors}
\begin{itemize}
\item A file descriptor is a non-negative integer, used to reference a file by
the kernel.
\item File descriptors 0, 1, and 2 are Standard Input, Standard Output, and
Standard Error respectively.
\end{itemize}

\subsection*{3.3 \code{open} and \code{openat} Functions}
\code{\#include <fctnl.h>}

\code{int open(const char *\emph{path}, int \emph{oflag}, \ldots \,/* mode\_t
\emph{mode} */ );}

\code{int openat(int \emph{fd}, const char *\emph{path}, int \emph{oflag},
\ldots\,/* mode\_t \emph{mode} */);}

Both return: file descriptors if OK, -1 on error

\begin{itemize}
\item The \emph{path} parameter is the name of the file to open or create.
\item The function has a multitude of options, which are specified by the
\emph{oflag} argument.
\item The file descriptor returned by \code{open} and \code{openat} is
guaranteed to be the lowest-numbered unused descriptor.
\item The \emph{fd} parameter for \code{openat} has three use cases:
\begin{enumerate}
\item The \emph{path} parameter specifies an absolute pathname. \emph{fd} is
ignored.
\item The \emph{path} parameter specifies a relative pathname and the \emph{fd}
parameter is a file descriptor that specifies the starting location in the file
system where the relative pathname is to be evaluated.
\item The \emph{path} parameter specifies a relative pathname and the \emph{fd}
parameter has the special value \code{AT\_FDCWD}.
\end{enumerate}
\item \code{openat} addresses two problems
\begin{enumerate}
\item It gives threads a way to use relative pathnames to open files in
directories other than the current working directory.
\item It provides a way to avoid time-of-check-to-time-of-use (TOCTTOU) errors
\end{enumerate}
\item TOCTTOU errors is the idea that a program is vulnerable if it makes two
file-based function calls where the second call depends on the results of the
first call. because the two calls are not atomic, the file can change between
the two calls, thereby invalidating the results of the first call.
\end{itemize}

\subsection*{3.4 \code{creat} Function}
\code{\#include <fcntl.h>}

\code{int creat(const char *\emph{path}, mode\_t \emph{mode});}
Returns: file descriptor opened for write-only if OK, -1 on error

\begin{itemize}
\item \code{creat} is equivalent to \code{open(\emph{path}, O\_WRONLY | O\_CREAT
| O\_TRUNC, \emph{mode});}
\item \code{creat} only opens a file for writing.
\end{itemize}

\subsection*{3.5 \code{close} Function}
\code{\#include <unistd.h>}

\code{int close(int \emph{fd});}
Returns: 0 if OK, -1 on error

\begin{itemize}
\item When a process terminates, all of its open files are closed automatically
by the kernel.
\end{itemize}

\subsection*{3.6 \code{lseek} Function}
\code{\#include <unistd.h>}

\code{off\_t lseek(int \emph{fd}, off\_t \emph{offset}, int \emph{whence});}

Returns: new file offset if OK, -1 on error

\begin{itemize}
\item The interpretation of the \emph{offset} depends on the value of the
\emph{whence} argument.
\begin{itemize}
\item If \emph{whence} is \code{SEEK\_SET}, the file's offset is set to
\emph{offset} bytes from the begging of the file.
\item If \emph{whence} is \code{SEEK\_CUR}, the file's offset is set to its
current value plus the \emph{offset}.
\item If \emph{whence} is \code{SEEK\_END}, the file's offset is set to the size
of the file plus the \emph{offset}.
\end{itemize}
\item A file's current offset must normally be a non-negative integer. However,
certain devices could allow negative offsets.
\item Thus, the return value from \code{lseek} should be tested against -1 if
there is an error.
\item \code{lseek} only records the current file offset within the kernel -- it
does not cause any I/O to take place.
\item If the offset is greater than the file's current size, the next
\code{write} call is said to \emph{extend} the file. This is also known as
creating a hole in a file and is allowed.
\item A hole in a file isn't required to have storage backing it on disk.
\end{itemize}

\subsection*{3.7 \code{read} Function}
\code{\#include <unistd.h>}

\code{ssize\_t read(int \emph{fd}, void *\emph{buf}, size\_t \emph{nbytes});}

Returns: number of bytes read, 0 if end of file, -1 on error

\begin{itemize}
\item There are several cases in which the number of bytes actually read is less
than the amount requested:
\begin{itemize}
\item When reading from a regular file, if the end of the file is reached before
the requested number of bytes have been read.
\item When reading from a terminal device.
\item When reading from a network.
\item When reading from a pipe or FIFO.
\item When reading from a record-oriented device.
\item When interrupted by a signal and a partial amount of data has already been
read.
\end{itemize}
\item The read operation starts at the file's current offset. Before a
successful return, the offset is incremented by the number of bytes actually
read.
\end{itemize}

\subsection*{3.8 \code{write} Function}
\code{\#include<unistd.h>}

\code{ssize\_t write(int \emph{fd}, const void *\emph{buf}, size\_t
\emph{nbytes});}

Returns: number of bytes written if OK, -1 on error

\begin{itemize}
\item A common cause for a \code{write} error is either filling up a disk or
exceeding the file size limit for a given process.
\item The write operation starts at the file's current offset.
\item If \code{O\_APPEND} option was specified when the file was opened, the
file's offset is set to the current end of the file before each write operation.
\item After a successful write, the file's offset is incremented by the number
of bytes actually written.
\end{itemize}

\subsection*{3.9 I/O Efficiency}
\begin{itemize}
\item When sequential reads are detected, the system tries to read in more data
than an application requests, assuming that the application will read it
shortly.
\end{itemize}

\subsection*{3.10 File Sharing}
\begin{itemize}
\item The UNIX System supports the sharing of open files among different
processes.
\item The kernel uses three data structures to represent an open file, and the
relationships among them determine the effect on process has on another with
regard to file sharing.
\begin{enumerate}
\item Every process has an entry in the process table. Within each process table
entry is a table of open file descriptors (a vector with one entry per
descriptor). Associated with each descriptor are
\begin{enumerate}
\item The file descriptor flags
\item A pointer to a file table entry
\end{enumerate}
\item The kernel maintains a file table for all open files. Each file table
entry contains
\begin{enumerate}
\item The file status flags for the file.
\item The current file offset
\item A pointer to the v-node table entry for the file.
\end{enumerate}
\item Each open file has a v-node structure that contains information about the
type of file and pointers to functions that operate on the file.
\end{enumerate}
\item Each process that opens a file gets its own file table entry, but only a
single v-node table entry is required for a given file. This is so each process
has its own current offset for the file.
\end{itemize}

\subsection*{3.11 Atomic Operations}
\subsubsection*{Appending to a File}
\begin{itemize}
\item Any operation that requires more than one function call cannot be atomic,
as there is always the possibility that the kernel might temporarily suspend the
process between the two function calls.
\end{itemize}

\subsubsection*{\code{pread} and \code{pwrite} Functions}
\code{\#include <unistd.h>}

\code{ssize\_t pread(int \emph{fd}, void *\emph{buf}, size\_t \emph{nbytes},
off\_t \emph{offset});}

Returns: number of bytes read, 0 if end of file, -1 on error

\code{ssize\_t pwrite(int \emph{fd}, const void *\emph{buf}, size\_t
\emph{nbytes}, off\_t \emph{offset});}

Returns: number of bytes written if OK, -1 on error

\begin{itemize}
\item Calling \code{pread} is equivalent to calling \code{lseek} followed by a
call to \code{read}, with the following exceptions.
\begin{itemize}
\item There is no way to interrupt the two operations that occur when we call
\code{pread}.
\item The current file offset is not updated.
\end{itemize}
\item Calling \code{pwrite} is equivalent to calling \code{lseek} followed by a
call to \code{write}.
\end{itemize}

\subsubsection*{Creating a File}
\begin{itemize}
\item In general, the term \emph{atomic operation} refers to an operation that
might be composed of multiple steps.
\item If the operation is performed atomically, either all the steps are
performed (on success) or none are performed (on failure).
\item It must not be possible for only a subset of the steps to be performed.
\end{itemize}

\subsection*{3.12 \code{dup} and \code{dup2} Functions}
\code{\#include <unistd.h>}

\code{int dup(int \emph{fd});}

\code{int dup2(int \emph{fd}, int \emph{fd2});}

Both return: new file descriptor if OK, -1 on error

\begin{itemize}
\item The new file descriptor returned by \code{dup} is guaranteed to be the
lowest-numbered available file descriptor.
\item With \code{dup2}, the value of the new file descriptor is specified with
the \code{dup2} argument.
\item The new file descriptor that is returned as the value of the functions
shares the same file table entry as the \emph{fd} argument.
\end{itemize}

\subsection*{3.13 \code{sync}, \code{fsync}, and \code{fdatasynch}
Functions}
\begin{itemize}
\item When we write data to a file, the data is normally copied by the kernel
into one of its buffers and queued for writing to disk at some later time
(\emph{delayed write}).
\item The kernel eventually writes all the delayed-write blocks to disk.
\item To ensure consistency of the file system with the buffers, we use the
\code{sync}, \code{fsync}, and \code{fdatasync} functions.
\end{itemize}

\code{\#include <unistd.h>}

\code{int fsync(int \emph{fd});}

\code{int fdatasync(int \emph{fd});}

Returns: 0 if OK, -1 on error\\

\code{void sync(void);}

\begin{itemize}
\item The function \code{sync} is called periodically from a system daemon,
often called \code{update}.
\item The function \code{fsync} refers only to a single file, specified by the
file descriptor \emph{fd}, and waits for the disk writes to complete before
returning.
\item The \code{fdatasync} function is similar to \code{fsync}, but it affects
only the data portions of a file.
\end{itemize}


\section*{Advanced Programming in the Unix Environment \\
Chapter 8: Process
Control}
\subsection*{8.2 Process Identifiers}
\begin{itemize}
\item Every process has a unique process ID, a non-negative integer.
\item As processes terminate, their IDs become candidates for reuse after some
time has passed.
\item Process ID 0 is usually the scheduler process and is often known as the
\emph{swapper}.
\item Process ID 1 is usually the \code{init} process and is invoked by the
kernel at the end of the bootstrap procedure.
\item \code{init} reads the system-dependent initialization files and brings the
system to a certain state.
\item \code{init} never dies and is a normal user process, not a system process
like the swapper.
\item The following functions return identifiers for processes
\end{itemize}

\code{\#include <unistd.h>}

\code{pid\_t getpid(void);}

Returns: process ID of calling process\\

\code{pid\_t getppid(void);}

Returns: parent process ID of calling process\\

\code{uid\_t getuid(void);}

Returns: real user ID of calling process\\

\code{uid\_t geteuid(void);}

Returns: effective user ID of calling process \\

\code{gid\_t getgid(void);}

Returns: real group ID of calling process\\

\code{gid\_t getegid(void);}

Returns: effective group ID of calling process

\begin{itemize}
\item Note that none of these functions has an error return.
\end{itemize}

\subsection*{8.3 \code{fork} Function}
\begin{itemize}
\item An existing process can create a new one by calling the \code{fork}
function.
\end{itemize}

\code{\#include <unistd.h>}

\code{pid\_t fork(void);}

Returns: 0 in child, process ID of child in parent

\begin{itemize}
\item The new process created by \code{fork} is called the \emph{child process}.
\item The function is called once but returns twice. The return value in the
child is 0, but the return value in the parent is the process ID of the new
child.
\item A process can have more than one child, and there is no function that
allows a process to obtain the process IDs of its children.
\item \code{fork} returns 0 to the child because a process can only have a
single parent.
\item The child can always call \code{getppid} to obtain the process ID of its
parent.
\item The child is a copy of the parent.
\item In general, we never know whether the child starts executing before the
parent or vice versa (this is determined by the scheduling algorithm used in the
kernel).
\end{itemize}

\subsubsection*{File Sharing}
\begin{itemize}
\item One characteristic of \code{fork} is that al file descriptors that are
open in teh parent are duplicated in the child.
\item It is important that the parent and the child share the same file offset.
\item If both parent and child write to the same descriptor, without any form of
synchronization, their output will be intermixed.
\item Besides the open files, numerous other properties of the parent are
inherited by the child:
\begin{itemize}
\item Real user ID, real group ID, effective user ID, and effective group ID
\item Supplementary group IDs
\item Process group ID
\item Session ID
\item Controlling terminal
\item The set-user-ID and set-group-ID flags
\item Current working directory
\item Root directory
\item File mode creation mask
\item Signal mask and dispositions
\item The close-on-exec flag for any open file descriptors
\item Environment
\item Attached shared memory segments
\item Memory mappings
\item Resource limits
\end{itemize}
\item The differences between the parent and child are
\begin{itemize}
\item The return values from \code{fork} are different
\item The process IDs are different
\item The two processes have different parent process IDs
\item The child's \code{tms\_utime}, \code{tms\_stime}, \code{tms\_cutime}, and
\code{tms\_cstime} values are set to 0.
\item File locks set by the parent are not inherited by the child.
\item Pending alarms are cleared for the child.
\item The set of pending signals for the child is set to the empty set.
\end{itemize}
\item The two main reasons for \code{fork} to fail are
\begin{enumerate}
\item if too many processes are already in the system
\item if the total number of processes for this real user ID excees the system's
limit
\end{enumerate}
\item When a process wants to duplicate itself so that the parent and the child
can each execute different sections of code at the same time.
\item When a process wants to execute a different program.
\end{enumerate}
\item A \code{fork} followed by an \code{exec} is called a \emph{spawn}.
\end{itemize}

\subsection*{8.4 \code{vfork} Function}
\begin{itemize}
\item The \code{vfork} function creates the new process without copying the
address space of the parent into the child. The child runs in the address space
of the parent until it calls either \code{exec} or \code{exit}.
\item \code{vfork} guarantees that the child runs first, until the child calls
\code{exec} or \code{exit}.
\end{itemize}

\subsection*{8.5 \code{exit} Functions}
\begin{itemize}
\item A process can terminate normally in five ways:
\begin{enumerate}
\item Executing a \code{return} from the \code{main} function.
\item Calling the \code{exit} function.
\item Calling the \code{\_exit} or \code{\_Exit} function.
\item Executing a \code{return} from the start routine of the last thread in the
process.
\item Calling the \code{pthread\_exit} function from the last thread in the
process.
\end{enumerate}
\item The three forms of abnormal termination are as follows:
\begin{enumerate}
\item Calling \code{abort}
\item When the process receives certain signals.
\item The last thread responds to a cancellation request.
\end{enumerate}
\item When a parent terminates before the child process terminates, the
\code{init} process becomes the parent process. In such a case, we say that the
process has been inherited by \code{init}.
\item This guarantees that every process has a parent.
\item The kernel keeps a small amount of information for every terminating
proces so that the information is available when the parent of the terminating
processs calls \code{wait} or \code{waitpid}.
\item This information consists of the process ID, the terminal status of the
process, and the amount of CPU time taken by the process.
\item A process that has terminated, but whose parent has not yet waited for it,
it called a \emph{zombie}.
\end{itemize}

\subsection*{8.6 \code{wait} and \code{waitpid} Functions}

\subsection*{8.7 \code{waitid} Function}

\subsection*{8.8 \code{wait3} and \code{wait4} Functions}

\subsection*{8.9 Race Conditions}

\subsection*{8.10 \code{exec} Functions}

\end{document}
