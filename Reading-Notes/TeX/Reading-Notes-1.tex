\documentclass[]{article}
\usepackage[all]{xy}
\usepackage{amsmath}
\usepackage{amssymb}
\usepackage{amsthm}
\usepackage{enumitem}
\usepackage{indentfirst}
\usepackage{listings}
\usepackage{multirow}
\usepackage{tikz}
\usepackage{tikz-qtree}
\usepackage{tipa}
\begin{document}

\newtheorem{thm}{Theorem}
\title{Operating Systems \\ COMS W4118 \\ Reading Notes for Lecture 1}
\author{Alexander Roth}
\date{2015 -- 01 -- 21}
\maketitle

% Operating System Concepts Essentials Second Edition
\section*{Operating System Concepts Essentials}

% Section 1.2 Computer-System Organization
\subsection*{Computer-System Organization}

% Subsection 1.2.1 Computer-System Operation
\subsubsection*{Computer-System Operation}
A modern general-purpose computer system consists of one or more CPUs and a
number of device controllers connected through a common bus that provides access
to shared memory.

For a computer to start running, it needs to have an initial program to run
(\textbf{bootstrap program}), which is typically stored in the computer
hardware. It must be able to locate the operating-system kernel and load it into
memory.

When the kernel is loaded into memory, certain processes begin as well. These
are known as \textbf{system processes} or \textbf{system daemons}.

\textbf{Interrupts} signal an occurrence of an event from either the software or
hardware. Software make \textbf{system calls} to trigger an interrupt.

Interrupts must transfer control to the appropriate interrupt service routines
and be handled quickly. Thus, a table of pointers to interrupt routines can be
used to provide the necessary speed. Interrupt routines are called indirectly
through the table. This table holds the addresses of the interrupt service
routines for various devices. This array (\textbf{interrupt vector}) of
addresses is then indexed by a unique device number, with the given interrupt
request, to provide the address of the interrupt service routine for the
interrupting device.

The interrupt architecture must also save the address of the interrupted
instruction.

% Subsection 1.2.2 Storage Structure
\subsubsection*{Storage Structure}
The CPU can load instructions only from memory, so any programs to run must be
stored there. Most general-purpose computers run most of their programs from
rewritable memory, called main memory or \textbf{random-access memory (RAM)}.

\textbf{Read-only memory (ROM)} cannot be changed, so only static programs such
as the bootstrap program are stored there.

All forms of memory provide an array of bytes. Each byte has its own address.
Interaction is achieved through a sequence of \texttt{load} and \texttt{store}
instructions to specific memory addresses.

A typical instruction-execution cycle, as executed on a system with a
\textbf{von Neumann architecutre}, first fetches an instruction from memory and
stores that instruction in the \textbf{instruction register}. The instruction is
then decoded and an operation is performed on it. The result may be stored back
in memory. The memory unit sees only a stream of memory addresses. It does not
know how they are generated or what they are.

Ideally the programs and data should reside in main memory permanently. However,
it is often not possible because:
\begin{enumerate}
\item Main memory is usually too small to store all needed programs and data
permanently.
\item Main memory is a \textbf{volatile} storage debice that loses its contents
when power is turned off or otherwise lost.
\end{enumerate}

Thus most computer systems provide \textbf{secondary storage} that is able to
hold large quantities of data permanently.

The wide variety of storage systems can be organized in a hierarchy according to
speed and cost. The higher levels are expensive, but they are fast. As we move
down the hierarchy, the cost per bit generally decreases, whereas the access
time generally increases.

\textbf{Volatile storage} loses its contents when the power to the device is
removed. Thus, data must be written to \textbf{nonvolatile storage} for
safekeeping.

The design of a complete memory system must use only as much expensive memory as
necessary while providing as much inexpensive, nonvolatile memory as possible.
Caches can be installed to improve performance where a large disparity in access
time or transfer rate exists between two components.

% Section 1.4 Operating-System Structure
\subsection*{Operating-System Structure}
An operating system provides the environment within which programs are executed.
(Duh)

\textbf{Multiprogramming} increases CPU utilization by organizing jobs (code and
data) so that the CPU always has one to execute.

The operating system keeps several jobs in memory simultaneously. The jobs are
kept initially on disk in the \textbf{job pool}. This pool consists of all
processes residing on disk awaiting allocation of main memory.

The set of jobs in memory can be a subset of the jobs kept in the job pool. The
operating system picks and begins to execute one of the jobs in memory.
Eventually, the job may have to wait for some task to complete. In a
multiprogrammed system, the operating system simply switches to, and executes,
another job. As long as at least one job needs to execute, the CPU is never
idle.

\textbf{Time sharing} (or \textbf{multitasking}) is a logical extension of
multiprogramming. In time-sharing systems, the CPU executes multiple jobs by
switching among them, but the switches occur so frequently that the users can
interact with each program while it is running.

Time sharing requires an \textbf{interactive} computer system, which provides
direct communication between the user and the system. The \textbf{response time}
should be short -- typically less than one second.

A time-shared operating system allows many users to share the computer
simultaneously. Since each action or command in a time-shared system tends to be
short, only a little CPU time is needed for each user.

A program loaded into memory is called a \textbf{process}. When a process
executes, it typically executes for only a short time before it either finishes
or needs to perform I/O.

In a time-sharing system, the operating system must ensure reasonable response
time. The goal is sometimes accomplished through \textbf{swapping}, whereby
processes are swapped in and out of main memory to the disk. A more common
method for ensuring reasonable response time is \textbf{virtual memory}, a
technique that allows the execution of a process that is not completely in
memory. This enables the users to run programs that are larger than actual
\textbf{physical memory}.

% Section 1.5 Operating-System Operations
\subsection*{Operating-System Operations}
Modern operating systems are \textbf{interrupt-driven}. Events are almost always
signaled by the occurrence of an interrupt or a trap. A \textbf{trap} (or an
\textbf{exception}) is a software-generated interrupt caused either by an error
or by a specific request from a user program.

% Subsection 1.5.1 Dual-Mode and Multimode Operation
\subsection*{Dual-Mode and Multimode Operation}
We need two separate \emph{modes} of operation: \textbf{user mode} and
\textbf{kernel mode} (\textbf{supervisor mode, system mode,} or
\textbf{privileged mode}). A bit, called the \textbf{mode bit}, is added to the
hardware of the computer to indicate the current mode:
\begin{description}
\item[0] kernel
\item[1] user
\end{description}
This bit allows us to distinguish between a task that is executed on behalf of
the user or the operating system.

Whenever the operating system gains control of the computer, it is in kernel
mode. The system always switches to user mode before passing control to a user
program.

The dual mode of operation provides us with the means for protecting the
operating system from errant users. We accomplish this by designating some of
the machine instructions that may cause harm as \textbf{privileged
instructions}. The hardware allows privileged instructions to be executed only
in kernel mode.

System calls provide the means for a user program to ask the operating system to
perform tasks reserved for the operating system on the user program's behalf.

When a system call is executed, it is typically treated by the hardware as a
software interrupt. Control passes through the interrupt vector to a service
routine in the operating system, and the mode bit is set to kernel mode.

% Subsection 1.5.2 Timer
\subsection*{Timer}
We cannot allow a user program to get stuck in an infinte loop or to fail to
call system services and never return control to the operating system.  Thus we
use a \textbf{timer}. A timer can be set to interrupt the computer after a
specified period. A \textbf{variable timer} is generally implemented by a fixed-
rate clock and a counter.

Before turning over control to the user, teh operating system ensures that the
timer is set to interrupt. If the timer interrupts, control transfers
automatically to the operating system, which may treat the interrupt as a fatal
error or may give the program more time.

% Subsubsection 1.8.3 Caching
\subsubsection*{Caching}
As informatin is used, it is copied into a faster storage system on a temporary
basis. We will look in this storage system first before proceeding. If it is not
in this cache, we will put a copy of it in the cache as we may need it soon.

Because caches have limited size, \textbf{cache management} is an important
design problem. Careful selection of the cache size and of a replacement policy
can result in greatly increased performance.

In a hierarchical storage structure, the same data may appear in different
levels of the storage system.

% Section 2.7 Operating-System Structure
\subsection*{Operating-System Structure}
A common approach is to partition the task into small components, or modules.
Each module should be a well-defined portion of the system, with carefully
defined inputs, outputs, and functions.

% Subsection 2.7.1 Simple Structure
\subsubsection*{Simple Structure}
Manuy operating systesm do not have well-defined structures. A monolithic
structure is difficult to implement and maintain; however, it has verly little
overhead in the system call interface or in communication within the kernel.

% Subsection 2.7.2 Layered Approach
\subsubsection*{Layered Approach}
The \textbf{layered approach} is when the operating system is broken ina number
of layers (levels). The bottom layer (level 0) is teh hardware; the highest
(layer $N$) is the user interface.

An operatying-system layer is an implementation of an abstract object made up of
data and the operations that can manipulate those data.

The layers are selected so that each uses functions and services of only lower-
level layers. If an error is found during the debugging of a particular layer,
the error must be on that layer, because the layers below it are already
debugged.

% Subsection 2.7.3 Microkernels
\subsubsection*{Microkernels}
This method structure the operating system by removing all nonessential
compoenents from the kernel and implementing them as system and user-level
programs.

The main function of the microkernel is to provide communication between the
client program and the various services that are also running in user space.
Communication is provided through \textbf{message passing}.

% Subsection 2.7.4 Modules
\subsubsection*{Modules}
\begin{description}
\item[Loadable kernel modules] The kernel has a set of core components and links
in additional services via modules, either at boot time or during run time.
\end{description}

Linking services dynamically is preferable to adding new features directly to
the kernel, which would require recompiling the kernel every time a change was
made.

% Subsection 2.7.5 Hybrid Systems
\subsubsection*{Hybrid Systems}
In practice, very few operating systems adopt a single, strictly defined
sturcture. Instead, they combine different structures, resulting in hybrid
systems that address performance, security, and usability issues.

% Section 2.10 System Boot
\subsection*{System Boot}
The procedure of starting a computer by loading the kernel is known as \textbf{booting} the system. On most computer systems, a small piece of code known as the \textbf{bootstrap program} or \textbf{bootstrap loader} locates the kernel.

\end{document}
